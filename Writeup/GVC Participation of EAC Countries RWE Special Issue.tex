\documentclass[a4paper]{article}

%% Language and font encodings
\usepackage[english]{babel}
\usepackage[utf8x]{inputenc}
\usepackage[T1]{fontenc}

%% Sets page size and margins
\usepackage[a4paper,top=3cm,bottom=2cm,left=3cm,right=3cm,marginparwidth=1.75cm]{geometry}

%% Other packages
\usepackage{amsmath}
\usepackage{graphicx}
\usepackage[colorinlistoftodos]{todonotes}
\usepackage[colorlinks=true, allcolors=blue]{hyperref}
\usepackage{apacite}
\AtBeginDocument{\urlstyle{APACsame}}
\usepackage[section]{placeins}
\usepackage{adjustbox}
\usepackage{natbib}
\usepackage{booktabs}
\addto\captionsenglish{\renewcommand*\contentsname{Table of Contents}}
\date{July 30, 2021}


\title{\textbf{Patterns of Regional and Global Value Chain Participation in the EAC}}
\author{Sebastian Krantz\footnote{ODI Fellow in the Macroeconomic Policy Department, Ministry of Finance, Planning and Economic Development of Uganda, 2020-2021. Previous academic affiliation: Graduate Institute of International and Development Studies (IHEID), Geneva. Academic affiliations from October 2021: Kiel Institute for the Global Economy (IFW) and Christian Albrechts University (CAU), Kiel. Permanent contact: \emph{sebastian.krantz@graduateinstitute.ch}.}}

\begin{document}
\maketitle

\begin{abstract}
Using global Multi-Region Input-Output (MRIO) data from 2005-2015, this paper empirically investigates the extent and patterns by which EAC countries have integrated into international production and Global Value Chains (GVCs), and the share of this integration accounted for by Regional Value Chains (RVCs). Results imply that the foreign content of exports (I2E) and the share of exports being re-exported (E2R) are stably between 10\% and 20\% in most EAC countries. Trade in intermediates with the rest of the world remains 12-14 times greater in value added (VA) terms than trade in intermediates inside the EAC. A significant development in the 2005-2015 period is only visible in %Rwanda and Tanzania which have increased their I2E (mostly from global suppliers), and in 
terms of Kenya becoming an important supplier of inputs to the EAC (higher E2R in EAC partners). In addition, a downstream shift is evident across EAC countries and sectors, by which more VA (both domestic and foreign) is used for the production of final goods, while maintaining high levels of exports in primary agriculture and mining. Regressions predicting VA at the sector-level suggest that higher I2E and E2R shares increase GDP with an average elasticity of $\geq 0.25$ in the course of 2 years. Estimates for manufacturing sectors were slightly higher at elasticities $\geq 0.3$ in response to E2R shifts. These results suggest that policy measures to increase EAC members integration into GVCs and RVCs, %such as easing NTBs, %, and to reverse the downstream trend, 
are likely to benefit EAC economic growth in the medium run.  
\end{abstract}

\vspace{4.8cm}



\newpage
%\tableofcontents
%\newpage
%\listoftables
%\listoffigures
%\newpage

\section{Introduction}

Global Value Chains (GVCs), referring to the quickly expanding internationalization of production networks, have become a central topic in trade and development policy. With the entry into force of AfCFTA in 2019 and progress towards its full enactment, as well as (re)negotiation of trade agreements between AfCFTA and the EU, the potential of a large common market in Africa for increased internationalization of production and GVC related trade, both within Africa and between Africa and the EU, are of great interest to research and policy alike. Towards the end of gauging the potential implications and distributional side-effects of AfCFTA for GVCs, it is, at least to some degree, instructive to study the effects of smaller efforts of regional integration and creation of a common market in Africa, as has been the case in East Africa with the East African Community (EAC). \newline 

(Re)founded in 2000 by Uganda, Kenya and Tanzania as body to facilitate regional cooperation, the EAC quickly became a vehicle for economic integration. A customs union became operational in January 2005, with Kenya, the region's largest exporter, continuing to pay duties on category B goods entering the other countries on a declining scale until 2010 (EAC Customs Union Protocol, Article 11 and \citet{aloo2017free}). Rwanda and Burundi acceded in 2007, joining the customs union in 2009. The customs union expanded to a common market for goods, labour, and capital effective in 2010. In 2013, the Protocol for the Establishment of the EAC Monetary Union was signed, aiming for monetary union within 10 years, subject to macro-fiscal convergence criteria. In 2016 the newly founded Republic of South Sudan joined the EAC, and the Democratic Republic of Congo joined in July 2022. Thus the EAC, particularly the years following the customs union in 2005 and the common market in 2010, provides a decent case study in terms of what AfCFTA aims to achieve on a larger scale.  \newline

\todo[inline]{Check all the information above again.}

While some work has been done on regional value chains (RVCs) in East Africa within specific industries, such as Maize value chains studied by \citet{daly2016maize}, there has not yet been a detailed exposition of the GVC participation of East African Countries using comprehensive data sources such as Inter-Country Input-Output (ICIO) tables and databases. A number of landmark studies have however been conducted regarding GVC integration of Africa and developing countries more broadly. \newline

One of the first comprehensive analyses of GVCs in Africa is provided by \citet{foster2015global}, using the EORA 25 sector database over the periods from 2000-2011. %\todo{See summary}. 
They find that Africa as a region is more involved in GVCs than many other developing regions, but much of the GVC involvement of Africa is in upstream production, and involves the supply of primary goods into production of final goods in other regions and countries. Downstream involvement in GVCs is relatively small, and shows little improvement in the 1995-2011 period. Furthermore, they find that there is a great deal of heterogeneity in GVC involvement across African countries, with a number of relatively successful countries that are heavily involved in GVCs and with a relatively large share of their involvement being in downstream GVCs. At a sectoral level, \citet{foster2015global} state that manufacturing and high-tech sectors are typically not major contributors to GVC participation in African countries. %While manufacturing tends to play a larger role in downstream involvement in GVCs, the agricultural sector still accounts for the largest part of downstream GVC involvement across African countries. %\newline % Changes over time in GVC participation tend to be driven by agriculture and services sectors, with high-tech services being particularly relevant for many African countries. Much of the change in GVC participation over time is driven by upstream production, with evidence to suggest that low-tech manufacturing has become less important in downstream production for a number of African countries. %\todo[inline]{shorten this}.
Inner-African GVCs are also found to be not particularly important for most African countries, with a number of exceptions in southern Africa. The EU tends to be the biggest GVC partner for Africa, with some evidence to suggest that the contributions of East and South-East Asia and transition countries are increasing. \newline


% Also on export sophistication: 
% Overall therefore, the results from the export sophistication measure provide mixed results. While developments for Africa as a whole have tended to be positive, Africa remains the region with the lowest values of this export sophistication index. Within Africa there have been a number of relative successes in terms of export sophistication development (examples being Benin, Madagascar, Mauritius and Tunisia), but a significant number of cases where export sophistication has regressed (the most notable examples being Mauritania, Sudan and Zimbabwe).

%The authors also compute various indicators on social upgrading indicating that a minority of African countries have been able to upgrade (prominent examples including Egypt, Nigeria and Tunisia). For most African countries the extent of upgrading is found too be lower than that for the average developing country. % Despite some evidence of upgrading across particular dimensions however, for most African countries we observe that the extent of upgrading tends to be lower than that for the average developing country. \newline
%Finally, the authors note that due to the overall low volume of exports in some countries, the importance  of GVCs may be overstated. \newline


% \citet{kwizera2019factors} provides a gravity analysis. 

%Most goods we use nowadays consist of parts that are sourced from different corners of the planet and are assembled across different continents. A popular example of this development is the iPhone, which uses in- puts from at least five countries (USA, China, Germany, Taiwan, South Korea) and is assembled in two (USA and China) \citep{Kummritz2014}. \newline

% \newline 
A broader research perspective for developing countries is provided in \citet{Kummritz20162} and \citet{Kummritz20161}. \citet{Kummritz20162} examine pattens of GVC integration in low-and middle income countries using the OECD TiVA database covering 61 countries and 34 industries for the years 1995, 2000, 2005, and 2008 to 2011. They find that, with exception of the agricultural sector, developing countries are typically located more downstream in the value chain, and export more final goods than high-income countries. They explain this by high-income economies using GVCs to outsource low value added downstream production stages and eventually reimport the final goods. Looking over time they find evidence suggesting that many developing economies have succeeded in moving up the value chain and that the general trend points to a more even distribution of value added across countries. Examining different regions, they find that South-East Asia has the highest levels of GVC integration, while Latin America and the Caribbean is more heterogenous with Chile and Costa Rica performing very well. In Africa, Tunisia has developed backward linkages into GVCs, especially with the EU. Overall their findings suggest that low- and middle-income countries have become an integral part of GVCs, where the foreign content of global value-added exports attributable to these countries has risen from 9\% in 1995 to 24\% in 2011, and the share in re-exported exports has increased from 9\% to 23\%. The authors stipulate that low- and middle-income countries are becoming drivers of GVC expansion and proceeding up the value chain to more upstream tasks, with beneficial effects for domestic industrialization. \newline
%f fva_fin in i2e has fallen by about 4\%. This gain accrues to the double counting part, which rises by 6\%. This means that production has become more fragmented and that developing economies increasingly occupy more upstream tasks

Another broad analysis of GVC participation focusing on Africa, the Middle East and Asia is provided by \citet{kowalski2015participation}. The authors conduct extensive econometric analysis using GVC indicators computed from OECD TiVA ICIOs for 57 countries in the years 1995, 2000, 2005, 2008 and 2008 (40 of which high-income), WIOD ICIOs for 40 (mostly high-income) countries from 1995-2011, and EORA ICIOs for 186 countries from 1990-2011, as well as country specific structural and policy indicators. They find that structural factors, such as geographic proximity to manufacturing hubs in Europe, North America and East Asia, the size of the domestic market and level of development are key determinants of GVC participation. In addition, trade and investment policy reforms (in particular low import tariffs and FDI openness), improvements of logistics and customs, intellectual property protection, infrastructure and institutions can play an large role in promoting further engagement. They state that very favourable policy environments in low-income countries can substitute to some extent for suboptimal structural factors. % \newline 
Their analysis also shows that important benefits arise for developing countries from GVC participation through both forward and backward linkages, including enhanced productivity, sophistication and diversification of exports. In analyzing export competitiveness of different regions, they find that Asian regions dominate the more advanced products such as electronic equipment or motor vehicles while African and Middle Eastern regions tend to be competitive in agriculture and food processing and in less advanced manufacturing products. %\newline 
While all regions have become more competitive, they find little signs for a trend towards industrialization and trade in intermediates in Africa. They also find that the survival rates of export relationships in Asia is nearly twice the survival rates of African export relations, which they attribute to stronger regional integration and learning by doing. In term of integation they find that advanced economies are becoming less important as suppliers of inputs to developing regions, and that trade in intermediates is increasing both between and within the South Asia, MENA and SSA regions. Finally, they find that services trade also takes on an increasing role in GVC developments in various developing regions. \newline 

This paper uses the EORA Global ICIO tables \citet{lenzen2012mapping, lenzen2013building}, aggregated to different world regions for non-EAC countries to analyze patterns of production in the EAC and compute standard GVC indicators for the years 2005-2015. The main aim of this research is to map the structure of regional and international production and exports in the EAC, and to produce some first evidence of the potential benefits of GVC integration for East Africa at the aggregate and sector levels. A secondary aim, more difficult to substantiate empirically, is to gauge the potential effects of regional economic integration through a customs union (2005) and common market (2010) for GVC related trade in and with the EAC.   %\newline
The analysis follows the seminal works of \citet{hummels2001nature}, \citet{koopman2014tracing}, \citet{wang2013quantifying}, as well as \citet{Kummritz20161} and \citet{Kummritz20162}. More sophisticated GVC decompositions and econometric approaches are considered infeasible in light of the low quality of ICIO tables for developing countries. 

\section{Data}
Most GVC analysis uses Inter-Country Input-Output tables (ICIOs), such as those
published by the OECD and WTO (TiVA) or the World Input Output Database (WIOD)  \citep{timmer2012world}. These tables state supply and demand relationships in gross terms between industries within and across countries \citep{Kummritz2014}. The former two databases are however limited to high-income or larger developing countries, with limited or no coverage of Sub-Saharan Africa. \newline

This research therefore uses the EORA Global ICIO tables \citep{lenzen2012mapping, lenzen2013building}, which have an extensive coverage of 189 countries, but rely on more sophisticated supercomputing methods to impute and harmonize data across countries and are therefore considered less reliable than the OECD or WIOD tables. % \newline
The EORA database comes in a Full version with heterogenous sector disaggregations as provided by country SUT tables, and an aggregated 26 sector version that is harmonized across countries. This research considers the EORA 26 databse, of which data until 2015 is available at the time of conducting this research\footnote{In 2021 an update of EORA was released, adding administrative data through 2018 and WEO based forecasts trough 2021 (which must be purchased). The revision however introduced a very large structural break into the time series in 2016, resulting in different macroeconomic totals and GVC indicators. It is not possible to obtain a complete revised series. Since this research is more concerned with the early years of EAC integration following the accession of Rwanda and Burundi in 2007, and with trends in GVC indicators, I decided to stick with the first edition of the EORA database with data through 2015. Appendix B shows some GVC indicators computed on the combined database, and discusses some implications of the structural break in the data. }. Since GVCs are a recent phenomenon, particularly in Africa, and the EAC customs union only became operational in 2005, with Rwanda and Burundi becoming full EAC members in 2007, I consider the EORA 26 tables from 2005-2015. \newline

To enhance the interpretation of results while preserving some level of detail about the non-EAC world, as well as reduce the strain on computational resources required to obtain results, the non-EAC countries are aggregated into 11 geographic and trade regions summarised in Table \ref{tab:ctry}. This reduces the size of the transaction tables from $189 \times 26 = 4914$ rows and columns to $(6 + 11)\times 26 = 442$ rows and columns. The 26 sectors are summarized in Table \ref{tab:sec}\footnote{Sector codes are assigned and used throughout the paper, but are not found in the EORA 26 database.}. \newline





\begin{table}[h!]
\centering
\caption{\textsc{Regions}}

\label{tab:ctry}
\vspace{2mm}
\begin{tabular}{llr} \toprule
\textit{Region} & \textit{Description} & \textit{Countries} \\ \midrule
EAC & East African Community & 6 \\
SSA & Sub-Saharan Africa (Excluding EAC) & 42 \\
EUU & European Union + UK & 28 \\
ECA & Europe and Central Asia (Non-EU) & 31 \\
MEA & Middle East and North Africa & 20 \\
NAC & North America and Canada & 3\\
LAC & Latin America and Carribean & 42 \\
ASE & ASEAN & 10 \\
SAS & South Asia & 8 \\
CHN & China & 3 \\
ROA & Rest of Asia & 11 \\
OCE & Oceania & 14
 \\ \bottomrule
\end{tabular}
\end{table}



\begin{table}[h!]
\centering
\caption{\textsc{Sectors}}

\label{tab:sec}
\vspace{2mm}
\begin{tabular}{ll} \toprule
\textit{Sector Code} & \textit{Description} \\ \midrule
AGR & Agriculture \\
 FIS & Fishing \\
 MIN & Mining and Quarrying \\
 FBE & Food \& Beverages \\
 TEX & Textiles and Wearing Apparel \\
 WAP & Wood and Paper \\
 PCM & Petroleum, Chemical and Non-Metallic Mineral Products \\
 MPR & Metal Products \\
 ELM & Electrical and Machinery \\
 TEQ & Transport Equipment \\
 MAN & Other Manufacturing \\
 REC & Recycling \\
 EGW & Electricity, Gas and Water \\
 CON & Construction \\
 MRE & Maintenance and Repair \\
 WTR & Wholesale Trade \\
 RTR & Retail Trade \\
 AFS & Hotels and Restraurants \\
 TRA & Transport \\
 PTE & Post and Telecommunications \\
 FIB & Finacial Intermediation and Business Activities \\
 PAD & Public Administration \\
 EHO & Education, Health and Other Services \\
 PHH & Private Households \\
 OTH & Others \\
 REI & Re-export \& Re-import \\ \bottomrule
\end{tabular}
\end{table}
\FloatBarrier


The values recorded in EORA are in thousands of current USD at basic prices\footnote{The basic price is the amount receivable by the producer from the purchaser for a unit of a good or service produced as output minus any tax payable, and plus any subsidy receivable, by the producer as a consequence of its production or sale. It excludes any transport charges invoiced separately by the producer.}. The data are scaled to be consistent with global aggregates, but may heavily distort data for smaller countries. The appendix shows measures of GDP and gross exports computed for the World and the EAC countries, indicating that global GDP is broadly consistent with representative estimates, but the GDP of EAC countries is highly distorted. Most notably, the GDP of Tanzania has been decreasing according to EORA data. The situation is better for exports, whose level and sectoral composition is at least broadly consistent with estimates from other sources. Thus the analysis and results presented below should be treated with extreme caution, particularly for Tanzania, as the data analyzed was not constructed to accurately reflect macroeconomic aggregates in developing countries. Nevertheless EORA is the only global ICIO database currently in existence and may be used to get at least a rough idea about production sharing and integration into Global Value Chains in the EAC. % Due to the severe shortcomings of the data in terms of macroeconomic representation of EAC countries, the emphasis is placed on analyzing broad trends in the data. 

\section{Gross Flows}
In light of the macroeconomic inconsistencies flagged above and the fact that value added flows are estimated from gross flows, it is useful to first consider the raw data in more detail before diving into detailed decompositions of trade flows. \newline

% \subsection{Intermediate Inputs}
An aggregated EORA 26 MRIO Table for the year 2015 is shown in Figure \ref{fig:wld}.  The columns of the table constitute production functions showing the intermediate inputs required by each of the column-countries or regions from each of the row-countries or regions to produce their output. Conversely the rows show quantities supplied by each row-country or region to each column-country or region. Flows are reported on a log10 scale due to their vastly different magnitudes. Among the EAC countries, the table shows a significant intermediate input supplier role of Kenya, supplying $10^{2.82} = 661$ million USD to Uganda, $10^{2.42} = 263$ million USD to Tanzania and  $10^{1.85} = 71$ million USD to Rwanda. Tanzania and Uganda appear to assume less of a supplier role, with Tanziania supplyzing 12 million USD to Uganda, 40 million to Kenya and 8 million to Rwanda, and Uganda supplying 8 million to Tanzania, 44 million to Kenya and 34 million to Rwanda. Rwanda appears to be insignificant in terms of it supplier role, supplying less than 1 million USD in inputs to any of its EAC partners. Burundi and South Sudan appear insignificant both as suppliers and consumers of intermediate inputs. With the rest of the world, Uganda, Tanzania and Kenya each import between 250 and 800 million USD if intermediate imports from the rest of Sub-Saharan Africa, and a similar magnitude from the Middle East, South Asia and China. The larges supplier of Inputs to each of the EAC countries appears to be the European Union supplying $10^{2.74} = 550$ million USD to Uganda, $10^{2.97} = 993$ million to Tanzania, $10^{3.44} = 2754$ million to Kenya, $10^{2.48} = 302$ million to Rwanda  $10^{1.96} = 91$ million to Burundi and $10^{1.11} = 13$ million to South Sudan. 
% \todo[inline]{Discuss EAC Supplier Roles as well, and perhaps percentages table??}.

\begin{figure}[h!]
\centering
\caption{\label{fig:wld}\textsc{Aggregated MRIO Table: EAC and World Regions}}
\small{\textit{Millions of 2015 USD at Basic Prices on a Log10 Scale}}
\includegraphics[width=1\textwidth, trim= {0 0 0 0}, clip]{"../Figures/heatmap_AG".pdf} %trim={<left> <lower> <right> <upper>}
\end{figure}
\FloatBarrier


To better understand the relative magnitude of IO relationships within the EAC vis-a-vis the rest of the world (ROW), Figure \ref{fig:GR} plots the relative magnitude of ROW inflows into EAC production to EAC inputs into EAC production (excluding own inputs), and next to it the ratio of EAC inputs into ROW production divided by EAC inputs into EAC production. 

\begin{figure}[h!]
\centering
\caption{\label{fig:GR}\textsc{Gross Flows Ratios: ROW/EAC Inflows and Outflows}}
% \small{\textit{Millions of 2015 USD at Basic Prices on a Log10 Scale}}
\includegraphics[width=1\textwidth, trim= {0 0 0 0}, clip]{"../Figures/GROSS_RATIOS".pdf} %trim={<left> <lower> <right> <upper>}
\end{figure}
\FloatBarrier

It is evident from Figure \ref{fig:GR} that the IO relationships of the EAC with the ROW have developed asymmetrically. ROW continues to supply 12 times more inputs for EAC production than other EAC members, but the ratio of EAC inputs for ROW production to EAC members inputs for EAC production has declined from 5.9 in 2005 to 4.4 in 2015. This implies a relatively greater demand for EAC inputs from EAC members compared to ROW, but an inability of EAC countries to increasingly meet the demands of their production with inputs sourced from EAC members. \newline


Figure \ref{fig:exp_EAC_share} shows the percentage of gross exports going to other EAC member countries and its sector-level breakdown. It is evident that Uganda and Kenya both have shares of 30\% of their exports to the EAC, and that for Uganda the largest part of these exports are agricultural, while for Kenya the largest part is manufacturing, in particular petro-chemicals, metal products and electric machinery. The other EAC members don't export very much to the EAC, in particular Rwanda, Burundi and South Sudan where the data suggests an EAC export share below 2\% in 2015. 

\begin{figure}[h!]
\centering
\caption{\label{fig:exp_EAC_share}\textsc{Percentage of Gross Exports Going to EAC Members}}
\includegraphics[width=1\textwidth, trim= {0 0 0 0}, clip]{"../Figures/exports_EAC_perc_stacked_ts".pdf} %trim={<left> <lower> <right> <upper>}
\end{figure}
\FloatBarrier


%\subsection{Gross Flows Decomposition}

% Note: This figure is presented as a stacked area chart in th eIntroduction Section. 
%\begin{figure}[h!]
%\centering
%\caption{\label{fig:outDVAtot}\textsc{Direct Value Added (GDP)}}
%\includegraphics[width=1\textwidth, trim= {0 0 0 0}, clip]{"../Figures/output_DVA_tot".pdf} %trim={<left> <lower> <right> <upper>}
%\end{figure}
%\FloatBarrier

%\begin{figure}[h!]
%\centering
%\caption{\label{fig:outDVA}\textsc{Direct Value Added Content of Output}}
%\includegraphics[width=1\textwidth, trim= {0 0 0 0}, clip]{"../Figures/output_DVA".pdf} %trim={<left> <lower> <right> <upper>}
%\end{figure}
%\FloatBarrier

Apart from exports to the EAC, we can also compute imports from other EAC countries alongside other metrics such as domestic VA, the share of imported inputs, and the share of output being exported. Figure \ref{fig:outshares_ag_ts} shows these shares over the 2005-2015 period. 'Value Added' gives the total share of domestic VA in output (VAS), which is at 50-60\% for all countries apart from Tanzania (due to inconsistencies in domestic data for Tanzania). The remainder of output (1-VAS) is comprised of domestic or imported intermediate goods. The 'Percent of Inputs Imported' gives the share of intermediate inputs that is imported. It is a gross measure of backward GVC integration, and less than 10\% in all EAC countries exempting Tanzania. Kenya appears to have the highest level at 10\% of inputs imported. Of the imported inputs, the 'Percent of Imports from EAC' shows the percentage coming from the EAC. It is a measure of backwards regional integration relative to the overall level of a countries international integration. %\newline  %Similarly, on the export side, Figure \ref{fig:outshares_ag_ts} reports both the overall percentage of output exported (i.e. not consumed domestically), and the percentage of exports going to the EAC neighbours. \newline
Here another asymmetry in EAC economic integration surfaces: whereas Uganda has both exports and import shares of 25-30\% with the EAC, Kenya only exports close to 30\%, with EAC share in imports at 2\%. Similarly, Rwanda imports 12-15\% from the EAC, but exports very little at $\sim$2\%. Other countries have both export and import shares below 10\%. Apart from Kenya and Uganda exporting more to the EAC over time, Figure \ref{fig:outshares_ag_ts} shows little aggregate development regarding EAC regional and global integration in gross terms. 

% Figure \ref{fig:outshares_ag_ts}
% The most curious finding presented by Figure \ref{fig:outshares_ag_ts} is the remarkable stability of shares, with few exceptions, suggesting only a very moderate increase in regional and global economic supply chains at the aggregate level. The starting levels of the countries are very different, with Uganda maintaining shares of exports and imports around 30\% with the EAC, whereas Tanzania, Rwanda and Burundi show much lower levels of integration. In Uganda and Kenya the percent of exports going to the EAC increased slightly over the sample period, while maintaining stable imported inputs and exported output shares. The overall increase in the percent of inputs imported in Tanzania may be due to a decline in domestic intermediates reflecting the decline in GDP, and should thus also be taken with extreme caution.  

% \todo[inline]{Put Value Added on the right axis.}

\begin{figure}[h!]
\centering
\caption{\label{fig:outshares_ag_ts}\textsc{Decomposition of Output and Exports}}
\includegraphics[width=1\textwidth, trim= {0 0 0 0}, clip]{"../Figures/output_shares_ag_ts".pdf} %trim={<left> <lower> <right> <upper>}
\end{figure}
\FloatBarrier


\section{Value Added Flows}
While gross flows provide useful information about direct productive relationships and the amounts of goods traded therein, they do not reveal how much of the value was added in the supplying industry, and how much of the value was added in previous stages of production performed by other industries or even countries. % \citep{Kummritz2014}. 
The Leontief decomposition of gross trade flows solves this problem by reallocating the value of intermediate goods used by industries to the original producers \citep{Kummritz2014}. %In our example, the use of Argentinian agricultural produce (raw hides) is subtracted from the Turkish leather industry and added to the Argentinian agricultural industry \citep{Kummritz2014}. 
%\newline

% \todo[inline]{Be less educational.}

\subsection{The Leontief Decomposition and Vertical Specialization}

Let $\textbf{A}$ be a row-normalized ICIO table where each element $a_{ij}$ gives the units of sector (row) $i$'s output required for the production of one unit of sector (column) $j$'s output, $\textbf{x}$ the vector of outputs of each country-industry and $\textbf{d}$ a vector of final demands such that the following productive relationship holds:

\begin{equation}
\textbf{x} = \textbf{A}\textbf{x} + \textbf{d}.
\end{equation}

The classical \citet{leontief1936quantitative} insight was that one can solve this equation for $\textbf{x}$ to get the amount of output each industry should produce given a certain amount of final demand:

\begin{equation} \label{eq:leontief}
\textbf{x} = (\textbf{I}-\textbf{A})^{-1} \textbf{d} = \textbf{B}\textbf{d},
\end{equation}

where the Leontief Inverse was denoted $\textbf{B} = (\textbf{I}-\textbf{A})^{-1}$. This matrix is also often called the total requirement matrix since it gives the total productive input requirement from each sector to produce one unit of final output\footnote{Specifically each element in $b_{ij}$ in \textbf{B} gives the output required from sector $i$ for the production  of one unit of the final good $j$. Thus the first column of \textbf{B} gives all the productive input required from all sectors for the production of one unit of the final good in sector 1, and the first row of \textbf{B} gives all the input required from sector 1 to produce one unit of the final good in each sector.}. Now the amount of direct value added in each unit of output for each sector is given by:

\begin{equation}
\textbf{v} = \textbf{1} - \textbf{A}'\textbf{1},
\end{equation}
where $\textbf{1} = (1, 1, 1, ..., 1)'$ is a column-vector of 1's\footnote{Thus the expression amounts to summing up the entries in each column of \textbf{A} (representing the intermediate input shares for 1 unit of output) and subtracting them from 1.}. Let \textbf{V} be the matrix with \textbf{v} along the diagonal and 0's in the off-diagonal elements. Multiplying Eq. \ref{eq:leontief} with $\textbf{V}$ therefore gives the value added in each sector:

\begin{equation} \label{eq:VB}
\textbf{V}\textbf{x} = \textbf{V}(\textbf{I}-\textbf{A})^{-1} \textbf{d} = \textbf{VBd}.
\end{equation}
The term $\textbf{VB} = \textbf{V}(\textbf{I}-\textbf{A})^{-1}$ is known as the matrix of value added multipliers or value added shares, which can be used to obtain the amount of value added generated in each industry (\textbf{Vx}) when producting to satisfy final demand (\textbf{d}). More specifically, the matrix $\textbf{VB}$ contains the amount of valued added by each sector (row) to the production of one unit of each sector's (column's) output. % \newline
%for column sector $j$ producing final output, these row sectors $i$ also require intermediate inputs as required by their own production (column). The matrix \textbf{AA} provides in each entry the units of intermediate input supplied by each sector $i$ for the generation of its own.... \newline
% Under the assumption that the production technology of goods is the same no matter whether they are domestically consumed or exported, the matrix \textbf{VB} to exports to obtain the value added origins thereof. \newline
The foreign value added share in domestic production and exports is what \citet{hummels2001nature} termed 'Vertical Specialization' (VS), and is the most widely used measure of backward GVC integration. Consider \textbf{VB} with elements vb$_{oi,dj}$ where $o$ is the VA origin country and $i$ the VA origin industry (along the rows) and $d$ is the VA using country and $j$ the VA using industry (along the columns). Then the VS ratio for a particular country-industry may be expressed as\footnote{In words: we are summing the elements of \textbf{VB} in each column, excluding any VA shares components from domestic country-industries.}

\begin{equation} \label{eq:VS}
\text{VS}_{uj} = \sum_{oi,\ o \neq  u} \text{vb}_{oi, uj}\ \ \forall uj.
\end{equation}

Figure \ref{fig:EACVB_ts} gives a breakdown of the VS for EAC countries by supplier country-region over the sample period. The share of foreign value added in Ugandan production has been fluctuating between 10 and 12\% over the analyzed period. 2\% of the value of Ugandan produce comes from Kenya, about 1\% from the rest of Sub-Saharan Africa, 1.5\% from the MENA region, about 3-4\% from the EU and 1-1.5\% from South Asia. The other regions make up the remaining 2\%. Tanzania and Kenya have similar relative VA contributions of SSA, MENA, EU and SAS regions to their production, at overall higher foreign content shares of around 16\% foreign VA in Kenya over the period. Data for Tanzania and Rwanda show stark trends, but these should be taken with caution, particularly given the declining GDP estimate of Tanzania (Figure \ref{fig:EAC_GDP_sec}), seems to encourage higher foreign VA to compensate for rising exports (Figure \ref{fig:exp}). Kenya adds around 1-2\% to Tanzanian production, and Uganda adds about 0.25\% to Kenyan produce. The value added share of Kenya and Uganda in Rwandan production of around and 1\% and 0.5\%, respectively is also visible. It is noteworthy that VS appears to have declined in Uganda, Kenya, Randa and Burundi from 2011 onwards. %, whereas in Tanzania it appears to have increased. % \todo{Why is that?}.

\begin{figure}[h!]
\centering
\caption{\label{fig:EACVB_ts}\textsc{Foreign Value Added Shares in EAC Production (VS)}}
\includegraphics[width=1\textwidth, trim= {0 0 0 0}, clip]{"../Figures/VA_shares_ag_ts_area".pdf} %trim={<left> <lower> <right> <upper>}
\end{figure}
\FloatBarrier

To better summarize the movements in VA shares by different origins over the analyzed period, Figure \ref{fig:EACVB_ts_bar} shows  bars giving the value added share in 2005 and in 2015, and above the two bars the annualized average growth rate in the share over this period. The annualized average growth rate of the domestic value added share is also reported in round brackets behind the country code for each EAC country. %Figure \ref{fig:EACVB_ts_bar} makes it clear that the foreign value added share has increased substantially in Tanzania and also in Rwanda, but decreased slightly in Kenya and Uganda between 2005 and 2015.  
Examining the inner EAC shares, it appears that Tanzania's VA share in Ugandan production has halved between 2005 and 2015, whereas Kenya's share has increased by 0.8\% annually. For Tanzania the VA share of Kenya has almost doubled from about 0.8\% to 1.8\%. Rwanda has also seen increases in the shares of Ugandan and Kenyan VA of about 5\% each year. In Burundi it appears foreign VA declined by 1.7\% over the sample period with only minor changes in the shares of different suppliers. The data for South Sudan is unreliable, also minding that the country only became independent in 2011. All of this indicates that Kenya is becoming an important supplier of inputs for EAC production, but that integration into regional and global value chains is stagnant otherwise. 

\begin{figure}[h!]
\centering
\caption{\label{fig:EACVB_ts_bar}\textsc{Change in Value Added Shares in EAC Production, 2005-2015}}
\includegraphics[width=1\textwidth, trim= {0 0 0 0}, clip]{"../Figures/VA_shares_ag_ts_bar".pdf} %trim={<left> <lower> <right> <upper>}
\end{figure}
\FloatBarrier



\subsection{Intermediate Inputs in Value Added Terms}

To examine productive relationships in VA terms, I use the matrix \textbf{VB} to decompose the transaction matrix \textbf{T} of gross IO flows into VA terms, to see who adds value in who's production.  

\begin{equation}
\textbf{T}^{VA} = \textbf{VB} \textbf{T}.
\end{equation} 

Table \ref{fig:VAwld} in Appendix C shows the resulting aggregate matrix, analogous to the gross flows shown in Figure \ref{fig:wld}. Compared to gross flows, foreign inputs to EAC members production, both from other EAC countries and from other world regions, are slightly greater in VA terms, while domestic VA content is lower. This indicates that more foreign VA is contained in domestic intermediates of EAC countries than domestic VA in foreign intermediates. \newline

Tables \ref{tab:VAweaclfl} and \ref{tab:VAeaclfl} show the largest flows involving EAC countries in VA terms. The ranking of the flows by magnitude is broadly similar in gross and VA terms, but it is evident in Tables \ref{tab:VAweaclfl} that Kenyan agricultural inputs to EU food processing industries are 30\% VA larger in VA terms, whereas EU inputs to Kenya, for example in Transport, are around 15\% smaller. This appears to give Kenya a more favorable position in VA intermediates trade with the EU, at least in terms of the main traded intermediate goods. A similar pattern can be observed on the right side of Table \ref{tab:VAweaclfl} where Kenya is excluded: incoming flows to EAC countries are between 20\% and 50\% smaller in VA terms whereas EAC outgoing flows are 10-30\% larger.  

% Table created by stargazer v.5.2.2 by Marek Hlavac, Harvard University. E-mail: hlavac at fas.harvard.edu
% Date and time: Tue, Jul 20, 2021 - 11:13:32 AM
\begin{table}[!htbp] \centering 
  \caption{\textsc{Largest Intermediate Flows Between the EAC and the World in VA Terms}} 
  \small{\textit{Millions of 2015 USD at Basic Prices}}
  \label{tab:VAweaclfl} 
  \vspace{2mm}
\begin{tabular}{rlrlr} \toprule
\textbf{\#} & \textbf{Flow} & \textbf{Value} & \textbf{Non-Kenya Flow} & \textbf{Value} \\ 
\midrule
1 & KEN.AGR $\to$  EUU.FBE & $597.299$ & EUU.ELM $\to$  TZA.ELM & $86.850$ \\ 
2 & KEN.AGR $\to$  EUU.REI & $287.130$ & EUU.ELM $\to$  UGA.ELM & $63.770$ \\ 
3 & MEA.TRA $\to$  KEN.TRA & $156.711$ & TZA.AGR $\to$  ROA.FBE & $61.865$ \\ 
4 & EUU.TRA $\to$  KEN.TRA & $151.987$ & UGA.AGR $\to$  EUU.FBE & $53.974$ \\ 
5 & EUU.FIB $\to$  KEN.CON & $120.347$ & EUU.FIB $\to$  TZA.FIB & $49.740$ \\ 
6 & EUU.ELM $\to$  KEN.CON & $119.281$ & EUU.FIB $\to$  TZA.ELM & $45.964$ \\ 
7 & KEN.AGR $\to$  EUU.AGR & $109.486$ & EUU.PCM $\to$  TZA.PCM & $44.280$ \\ 
8 & EUU.PCM $\to$  KEN.CON & $100.395$ & TZA.AGR $\to$  EUU.FBE & $43.597$ \\ 
9 & OCE.AGR $\to$  KEN.FBE & $94.364$ & MEA.ELM $\to$  UGA.ELM & $38.484$ \\ 
10 & EUU.PCM $\to$  KEN.PCM & $94.076$ & EUU.ELM $\to$  TZA.TEQ & $35.596$ \\ 
11 & EUU.FIB $\to$  KEN.TRA & $93.883$ & EUU.FIB $\to$  UGA.ELM & $33.491$ \\ 
12 & KEN.AGR $\to$  EUU.AFS & $91.138$ & EUU.FIB $\to$  UGA.FIB & $32.685$ \\ 
13 & EUU.ELM $\to$  TZA.ELM & $86.850$ & EUU.ELM $\to$  TZA.FIB & $31.445$ \\ 
14 & EUU.FIB $\to$  KEN.FBE & $83.665$ & SSA.FIB $\to$  TZA.FIB & $31.237$ \\ 
15 & EUU.FIB $\to$  KEN.PCM & $79.533$ & EUU.FIB $\to$  TZA.EHO & $30.858$ \\ 
16 & EUU.PCM $\to$  KEN.FBE & $76.573$ & EUU.FIB $\to$  TZA.PCM & $29.350$ \\ 
17 & KEN.FBE $\to$  EUU.FBE & $76.269$ & ROA.WTR $\to$  TZA.WTR & $29.330$ \\ 
18 & KEN.AGR $\to$  EUU.PCM & $74.543$ & SAS.PCM $\to$  TZA.PCM & $29.056$ \\ 
19 & EUU.ELM $\to$  KEN.ELM & $73.309$ & SSA.ELM $\to$  TZA.ELM & $28.566$ \\ 
20 & EUU.PCM $\to$  KEN.AGR & $72.236$ & EUU.MPR $\to$  TZA.ELM & $28.528$ \\ 
\bottomrule
\end{tabular} 
\end{table} 
\FloatBarrier

The discrepancy between gross flows and VA flows observed here is mostly explained by the nature of products traded between the EAC and ROW: The EAC mostly exports agricultural and other primary products with high domestic VA, whereas the EU and others mostly export manufactured goods and other complex products to the EAC, the creation of which involves imports of intermediates, and thus lower domestic VA. The result is that gross IO flows understate the relative importance of the EAC as a supplier of inputs to ROW. This finding is summarized by Figure \ref{fig:TBint}, which shows the trade balance in intermediate goods, expressed as the ratio of intermediate exports to intermediate imports, between the EAC and the rest of the world, in gross and VA terms. It is evident that the value of intermediates supplied to the EAC is more than twice

\begin{figure}[h!]
\centering
\caption{\label{fig:TBint}\textsc{EAC Trade Balance in Intermediate Goods in Gross and VA Terms}}
% \small{\textit{Millions of 2015 USD at Basic Prices on a Log10 Scale}}
\includegraphics[width=0.8\textwidth, trim= {0 0 0 0}, clip]{"../Figures/TB_INTER".pdf} %trim={<left> <lower> <right> <upper>}
\end{figure}
\FloatBarrier

\noindent as large than the value of intermediates supplied by the EAC, and this ratio has been deteriorating between 2005 and 2013 so that gross intermediate imports in 2013 were 3 times larger than gross intermediate exports. It is however also evident in Figure \ref{fig:TBint} that this ratio in VA terms is consistently larger by a value of 0.03. For example in 2015 EAC intermediate exports were 36\% of intermediate imports in gross terms, but 39\% of intermediate imports in VA terms. \newline

Table \ref{tab:VAeaclfl} shows the 20 largest intermediate flows within the EAC in VA terms. Compared to Table \ref{tab:eaclfl}, these flows are smaller by 10-60\%, which is due to the dominance of manufacturing inputs with lower domestic VA shares. In some cases the difference can be quite significant, for example inputs from the Ugandan to the Rwandan petro-chemical industry in 2015 were 2.54 million in gross terms, but 1.17 in VA terms. Primary flows increase in importance when considering VA terms, e.g. Ugandan agricultural inputs to Kenyan food processing industries are the 4th largest inner-EAC flow in VA terms, but only the 6th largest in gross terms. 

% Table created by stargazer v.5.2.2 by Marek Hlavac, Harvard University. E-mail: hlavac at fas.harvard.edu
% Date and time: Tue, Jul 20, 2021 - 11:12:50 AM
\begin{table}[!htbp] \vspace{-2mm}
  \centering 
  \caption{\textsc{Largest Inter-Country Intermediate Flows within the EAC in VA Terms}} 
  \small{\textit{Millions of 2015 USD at Basic Prices}}
  \label{tab:VAeaclfl} 
  \vspace{2mm}
\begin{tabular}{rlrlr} \toprule
\textbf{\#} & \textbf{Flow} & \textbf{Value} & \textbf{Non-Kenya Flow} & \textbf{Value} \\ 
\midrule
1 & KEN.MIN $\to$  UGA.PCM & $80.599$ & UGA.TRA $\to$  RWA.PAD & $1.720$ \\ 
2 & KEN.PCM $\to$  UGA.PCM & $45.927$ & UGA.TRA $\to$  RWA.TRA & $1.366$ \\ 
3 & KEN.PCM $\to$  TZA.PCM & $23.150$ & UGA.AGR $\to$  RWA.FBE & $1.215$ \\ 
4 & UGA.AGR $\to$  KEN.FBE & $22.839$ & UGA.MPR $\to$  RWA.MPR & $1.181$ \\ 
5 & KEN.WAP $\to$  UGA.WAP & $20.393$ & UGA.PCM $\to$  RWA.PCM & $1.170$ \\ 
6 & KEN.ELM $\to$  UGA.ELM & $19.860$ & UGA.MPR $\to$  RWA.ELM & $1.109$ \\ 
7 & KEN.PCM $\to$  UGA.EHO & $18.594$ & UGA.FIB $\to$  RWA.ELM & $1.054$ \\ 
8 & KEN.FIB $\to$  UGA.PCM & $16.246$ & UGA.FIB $\to$  RWA.PCM & $1.007$ \\ 
9 & KEN.MIN $\to$  TZA.PCM & $14.535$ & UGA.FIB $\to$  RWA.PAD & $0.885$ \\ 
10 & KEN.MIN $\to$  UGA.EGW & $14.426$ & UGA.FIB $\to$  RWA.FIB & $0.851$ \\ 
11 & KEN.PCM $\to$  UGA.CON & $13.471$ & UGA.WTR $\to$  RWA.WTR & $0.796$ \\ 
12 & KEN.AGR $\to$  UGA.FBE & $13.217$ & UGA.FIB $\to$  RWA.EHO & $0.787$ \\ 
13 & KEN.MIN $\to$  UGA.EHO & $12.528$ & UGA.ELM $\to$  RWA.ELM & $0.733$ \\ 
14 & KEN.TRA $\to$  UGA.PAD & $12.271$ & UGA.FIB $\to$  RWA.FBE & $0.665$ \\ 
15 & KEN.FIB $\to$  UGA.FIB & $12.121$ & UGA.TRA $\to$  RWA.FIB & $0.665$ \\ 
16 & KEN.MIN $\to$  UGA.CON & $11.871$ & TZA.MIN $\to$  UGA.PCM & $0.659$ \\ 
17 & KEN.PCM $\to$  UGA.FIB & $11.832$ & TZA.FIB $\to$  UGA.FIB & $0.658$ \\ 
18 & KEN.WAP $\to$  UGA.CON & $11.505$ & UGA.FIB $\to$  RWA.MPR & $0.646$ \\ 
19 & KEN.TRA $\to$  UGA.TRA & $10.848$ & UGA.FIB $\to$  RWA.CON & $0.645$ \\ 
20 & KEN.MPR $\to$  UGA.ELM & $10.620$ & UGA.FIB $\to$  RWA.TRA & $0.579$ \\ 
\bottomrule
\end{tabular} 
\end{table} 
\FloatBarrier

Having found that both ROW inputs to the EAC and inner-EAC flows, being dominated by manufacturing inputs, are smaller in VA terms, it remains to assess their relative importance in VA terms.  Figure \ref{fig:VAR} shows the ratio of ROW VA flows to the EAC to inner-EAC VA flows (excl.

\begin{figure}[h!] \vspace{-2mm}
\centering
\caption{\label{fig:VAR}\textsc{VA Flows Ratios: ROW/EAC Inflows and Outflows}}
% \small{\textit{Millions of 2015 USD at Basic Prices on a Log10 Scale}}
\includegraphics[width=1\textwidth, trim= {0 0 0 0}, clip]{"../Figures/VA_RATIOS".pdf} %trim={<left> <lower> <right> <upper>}
\vspace{-15mm}
\end{figure}
\FloatBarrier

\noindent domestic flows), analogous to Figure \ref{fig:GR}.  This ratio is higher in VA terms than in gross terms, with ROW providing on average 14 times more VA in EAC countries production than the EAC neighbours. In terms of outflows, a similar development as in Figure \ref{fig:GR} is evident, with the relative importance of ROW declining from 7.4 times greater in 2005 to 5.5 times greater in 2015.  \newline

Thus in summary, although the EAC is more important to ROW in VA terms, as a supplier of (mostly primary) inputs, the relative magnitude of ROW inflows to the EAC compared to intermediate flows inside the EAC is greater in VA terms than in gross terms. Thus in VA terms trade in intermediates inside the EAC plays only a minor role compared to productive relationships between EAC members and ROW. 



\subsection{Value Added Exports}

The matrix \textbf{VB}, reflecting the structure of international production, allows us to decompose any gross flow coming from any country-industry into it's value-added origins by country-industry. 

Next to VS or the imported content of production and exports visualized in detail in Figure \ref{fig:EACVB_ts} which functions as a measure of backward GVC integration, \citet{hummels2001nature} also introduced the share of domestic exports that enters foreign countries exports, which they called VS1, as a measure of forward GVC Integration. This measure was first computed and explored by \citet{daudin2011produces}. It is defined as 

\begin{equation} \label{eq:VS1}
\text{VS1}_{oi} = \frac{1}{E_{oi}} \sum_{uj, u \neq  o} \text{vbe}_{oi, uj}\ \ \forall\ oi,
\end{equation}

where $E_{oi}$ are the gross exports of country-industry $oi$ which is used to normalize the sum along the rows of \textbf{VBE} (excluding domestic industries) which capture the use of VA from a domestic sector $oi$ in the exports of all foreign sectors $uj$. For completeness I note that VS can be defined in an analogous way as

\begin{equation} \label{eq:VSE}
\text{VS}_{uj} = \frac{1}{E_{uj}} \sum_{oi, o \neq  u} \text{vbe}_{oi, uj}\ \ \forall\ uj,
\end{equation}

however since $\sum_{oi} \text{vb}_{oi, uj} = 1\ \forall\ uj$, the exports cancel out and Eq. \ref{eq:VSE} reduces to Eq. \ref{eq:VS}. \newline

Figure \ref{fig:VSag_ts} shows the aggregate VS and VS1 for each EAC member country over time. VS1 is called E2R (export to re-exports), and VS I2E (import to exports) by \citet{Kummritz20162} (following \citet{baldwin2015supply}), which also developed the $gvc$ R package to compute these measures. For Uganda, both VS and VS1 are at 11-12\% of exports towards the end of 2015, whereas VS1 was higher at around 14\% of exports in 2005. Tanzania, as already noted, shows a remarkable increase in backward GVC participation to above 30\% of VA in its produce generated abroad, but a slight decline in forward GVC participation down to 11\% of exports being re-exported in 2015 - similar to Uganda. Kenya exhibits a stable development with VS of around 17\% and VS1 around 12.5\%. Rwanda increased in VS from 15\% in 2005 to 22.5\% in 2015, and at the same time showed a decrease in VS1 from initially 22.5\% down to 16\% in 2015. It is noteworthy that while a few members like Tanzania and Rwanda appear to have succesfully increased their use of foreign intermediates, none of the members significantly increased it's role as supplier of inputs in the global market. Overall the GVC situation appears stagnant in Uganda, Kenya and Burundi. 

% \todo[inline]{Global i2e (VS) share is taken by \citep{Kummritz20162} as measure of increase in GVCs over time: "the nomial value of i2e has grown by approximately 350\% and as a share of total exports, it has grown by 35\%, from around 17\% to over 23\% of total exports, Thus, countries increasingly rely on inputs produced abroad for their export production."}. 

%\todo[inline]{Also from \citep{Kummritz20162}: Another way to examine the expansion of GVCs from 1995 to 2011 is to look at their length instead of their trade volume. WWZ propose to use the amount of double counted trade, pdc, as a proxy for GVC length, since its value goes up with back-and-forth trade, which is equivalent to an increase in the number of production stages. They show that its value has increased for 40 selected countries. "In Figure 2, we observe in our larger sample similarly that pdc as a share of total exports has increased over the examined period by 73\% and thus more than i2e. Therefore, GVCs do not only channel more trade but also have become longer over time."}


\begin{figure}[h!]
\centering
\caption{\label{fig:VSag_ts}\textsc{GVC Integration of EAC Members: Aggregate}}
\includegraphics[width=1\textwidth, trim= {0 0 0 0}, clip]{"../Figures/VS_ag_ts".pdf} %trim={<left> <lower> <right> <upper>}
\end{figure}
\FloatBarrier

%\todo[inline]{Compare growth of i2e and e2r to world growth in these ratios. Also: Is VS1 the same as E2R?}

%\begin{figure}[h!]
%\centering
%\caption{\label{fig:VSag}\textsc{Vertical Specialization: Aggregate}}
%\includegraphics[width=1\textwidth, trim= {0 0 0 0}, clip]{"../Figures/VS_aggregate".pdf} %trim={<left> <lower> <right> <upper>}
%\end{figure}
%\FloatBarrier


% \todo[inline]{Compute the previous 2 figures for the EAC only!}

% \todo[inline]{TODO: Do trade balance (ratios) in gross and VA terms for EAC members, and also state the relative importance of intermediate and final goods trade, both within EAC and between EAC and ROW.}


\subsection{Regional Integration in Value Added Trade}

So far we have explored standard country and industry level metrics of backward and forward GVC integration, VS and VS1, and examined the overall level of GVC integration of EAC countries and sectors. The result of this examination was that apart from increases in the import content of production and exports (VS) in Tanzania and Rwanda, the situation is relatively stable over time, with moderate amounts of heterogeneity at the sector level. \newline

This section builds on these findings and introduces some metrics to track regional EAC integration through VA in supply chains, relative to any global developments experienced by each member country. The first such metric computed is the share of foreign VA in a members production / exports accounted for by it's EAC partner states. It is computed as

\begin{equation} \label{eq:VS_EAC}
\text{VS}_{uj}^{EAC} = \frac{1}{\text{VS}_{uj}}  \sum_{oi \in EAC,\ o \neq  u} \text{vb}_{oi, uj}   \ \ \forall\ uj \in EAC,
\end{equation}

where VS$_{uj}$ is defined as in Eq. \ref{eq:VS}. VS$^{EAC}$ is thus a relative measure tracking the EAC share in VS, such that the overall EAC VA share in domestic production can be computed as VS$_{uj}^{EAC} \times \text{VS}_{uj} \ \forall\ uj$. We can define an analogous measure for VS1 as the proportion of domestic VA in re-exported exports that is exported by EAC partner states. 

\begin{equation} \label{eq:VS1_EAC}
\text{VS1}_{oi}^{EAC} =  \sum_{uj \in EAC, u \neq  o} \text{vbe}_{oi, uj} \bigg/ \sum_{uj, u \neq  o} \text{vbe}_{oi, uj}\ \ \forall\ oi \in EAC.
\end{equation}

These two metrics effectively track the role of the EAC in forming the interaction of each member country with the rest of the world in terms of production and export linkages. They do however not account for the import side, that is the overall role of the EAC in providing goods and services to each member country relative to the rest of the world. Therefore we will compute two additional metrics to capture this aspect of regional integration. The first metric is the share of EAC VA in the imports received by each member, which we shall denote by VAI$^{EAC}$. Consider $E_u$ to the vector of VA exports to EAC using country $u \in EAC$ from each country-sector\footnote{Note that since we don't have final demand disaggregated by sector, we cannot compute VAI$^{EAC}$ by receiving sector, but only by receiving country.}.%, and $\textbf{I}_u$ the matrix with  $I_u$ along the diagonal.
Then we can find the VA origins of these exports to country $u$ by pre-multiplying with \textbf{VB} to give 
\begin{equation}
E_u^{VA} = \textbf{VB}E_u,
\end{equation}
 where $E_u^{VA}$ denotes the vector, with elements $e_{oi, u}^{VA}$, of VA supplied by each country-industry ($oi$) in these imports of country $u$. From  $E_u^{VA}$ the share of EAC VA is easily computed as 
\begin{equation}
\text{VAI}_u^{EAC} = \sum_{oi \in EAC, o \neq u}  e_{oi, u}^{VA}  \bigg/ \sum_{oi, o \neq u}  e_{oi, u}^{VA}.  
\end{equation}
VAI$_u^{EAC}$ thus gives us a country-level measure of the VA by it's EAC partners in it's import mix, excluding any domestic VA in imports. This VA may however enter into produced goods and be exported again, thus it is a measure of the EAC contribution to (non-domestic) production and consumption in a particular member country. To single out the EAC share in imported consumption goods, we need to consider only exports for final demand, which exclude exports feeding into production as intermediates. Let $FE_u$ therefore denote the final exports to country $u$ from each country-industry. Then $FE_u^{VA} = \textbf{VB}FE_u$ denotes the decomposition of those exports into VA components, and we can define 
\begin{equation}
\text{VAFI}_{u}^{EAC} = \sum_{oi \in EAC, o \neq u}  fe_{oi, u}^{VA}  \bigg/ \sum_{oi, o \neq u}  fe_{oi, u}^{VA}
\end{equation}
as the EAC VA share in final goods exported to a particular member $u$. 

%\begin{equation} \label{eq:VS1_EAC}
%VS1_{oi}^{EAC} =  \sum_{uj \in EAC, u \neq  o} vbe_{oi, uj} \bigg/ \sum_{uj, u \neq  o} vbe_{oi, uj}\ \ \forall\ oi \in EAC.
%\end{equation}
%
%
%Other interesting metrics for EAC interegation are to consider the share of foreign value added in EAC exports coming from EAC partner states, as well as the share of EAC value added in final goods exports to the EAC, shown in Figure \ref{fig:VAFDexpEACshares}. 
%
%
%The EAC\_FVAX metric shows the VA percentage  share of other EAC countries in the exports of a particular member country, whereas the FVAX\_EAC metric shows the percentage of VA from EAC countries in the exports to the respective country. Both metrics exclude exports that are ultimately consumed in the country of origin \todo{Does this therefore also exclude domestic VA?? Also Interpret graph still..}. Using the notation introduced for Eq. \ref{eq:VS}, we can define EAC\_FVAX for each country as
%
%\begin{equation} \label{eq:VS_EAC}
%VS_{uj}^{EAC} = \frac{1}{VS_{uj}}  \sum_{oi \in EAC,\ o \neq  u} vb_{oi, uj}   \ \ \forall uj \in EAC,
%\end{equation}

\begin{figure}[h!]
\centering
\caption{\label{fig:VAEACshares}\textsc{EAC VA Shares in Members VS, VS1, Imports and Final Imports}}
\includegraphics[width=1\textwidth, trim= {0 0 0 0}, clip]{"../Figures/VA_EAC_shares_ts".pdf} %trim={<left> <lower> <right> <upper>}
\end{figure}
\FloatBarrier

Figure \ref{fig:VAEACshares} reports the results. It is evident that Uganda and Tanzania follow very similar regional integration patterns, though at very different levels. In Uganda, around 21\% of VS (the foreign content of production) is accounted for by the EAC, whereas in Tanzania this was 6.3\% at the end of 2015. In Uganda the EAC share of VS1 (re-exported exports) was close to 6\% end of 2015, whereas in Tanzania it ended at 2\%. In-between we have the EAC share in Ugandan imports (VAI) and final imports (VAFI) at around 16.5\% and 14.5\% in 2015, whereas for Tanzania these shares were 5.2\% and 4.5\%, respectively. \newline

% \todo[inline]{Another metric: Calculate each countries share in total EAC exports ??, or add EAC content in Final exports ??}

Overall this suggests that both countries have stronger backward GVC linkages with the EAC, with EAC countries (in particularly Kenya), supplying inputs into the production, whereas both countries play only a moderate role as suppliers of intermediates for export. In Kenya we observe the opposite pattern, where around 7.4\% of Kenya's re-exported exports (VS1) are exported by it's EAC partners, but only about 0.9\% of its imported inputs (VS) is accounted for by EAC partners. On the import side, less than 3\% of the VA in Kenyan imports is generated by it's EAC partners. It is interesting that the EAC share in final imports is higher at 2.3\% than the EAC share in overall imports at 1.5\%, confirming that Kenya imports relatively more final goods than intermediates from it's EAC partners. Rwanda and Burundi also follow a similar pattern of EAC integration. In both countries the final import share of the EAC is highest, at around 11\% in Rwanda and 4.3\% in Burundi in 2015. This is followed, with some distance, by the EAC share in VS, at 6.5\% in Rwanda and 2.2\% in Burundi. Both countries have a negligible supplier role for the EAC, with <1\% of their re-exported exports exported by EAC partners. In South Sudan the data suggest EAC shares in VS, VAI and VAFI of around 1\% in 2015, indicative of it low-level of economic integration with its EAC partners. Overall, the progression of these indicators over time suggests a stagnant or, in the case of Tanzania and Burundi, even declining level of regional integration through value chains, measured in relation to an also mostly stagnant overall level of GVC integration of the EAC member countries. 


% \todo[inline]{Compare direct domestic VA with total domestic VA}.
% \todo[inline]{Growth of VS relative to World Average, and percentage of export growth accounted for by VS growth, following Hummels. Also calculate VS trends in time. Also look at which industries are driving this growth.... ALso: VS growth accounted for bz regional EAC integration vs. other. ALso: Links between VS and FDI. And: VS and tarriffs / NTB's...}

% \todo[inline]{look at VS by firm size and owndership in each sector (combine ES with GVC data). Could also be interesting for economic integration in Uganda / East Africa. I.e. does the domestic economy benefit at all or is it just foreign firms assembling foreign inputs here.}

% \todo[inline]{KWW talk about effect of Exchange rate on bilateral trade atters and how it depends on VS -> understand it...}.

\subsection{Koopman Wang Wei Decomposition of Gross Exports}
One problem with the Leontief decomposition of gross exports into VA origins is that it also captures so called pure double counted items, which are items that that are traded two or more times between the same trading partners. For example if on a value chain for chemical products an intermediate product would be first exported from Uganda to Kenya, processed further in Kenya and then imported again by Uganda to produce a final good that is the exported. The Leontief decomposition will correctly allocate the share of VA in this product to Uganda and Kenya, but Ugandan gross exports themselves would overstate the amount of VA in either of the two countries, because it includes both the export of the intermediate produce to Kenya, and the export of the final good to the rest of the world. So this kind of double counting in gross exports is incurred whenever there exists two-way trade in intermediate goods. \newline

Secondly, the Leontief decomposition provides no information as to where and how the VA in exports is absorbed, it only provides the origin of VA in gross exports. To account for double counted items in gross exports and also to better understand where and how VA is absorbed, which indicates how countries integrate into GVCs, a number of increasingly complex GVC decompositions have been developed. The simplest and most well known of these is the decomposition of country-level gross exports into 9 VA components proposed by Koopman, Wang and Wei (2014), henceforth KWW \citep{koopman2014tracing}. The 9 terms fo the KWW decompostion are given schematically in Figure \ref{fig:KWW}\footnote{A mathematical expression for each of the 9 terms is provided in Eq. 36 of the \citet{koopman2014tracing} AER paper.}. \newline

The decomposition splits exports first into foreign content (VS) and domestic content. Domestic content is then further split into VA exports that are absorbed abroad, and content that eventually returns home and is absorbed domestically. Each of these thee categories can be subdivided further according to how the VA is utilized. In both the domestic content returning home and foreign content, there are double counted categories, which split double counted VA arising from two-way trade in intermediate goods according to their domestic and foreign VA. In the example given above: if an intermediate is first exported from Uganda to Kenya, then re-imported by Uganda and finally exported, then the VA in that first export from Uganda to Kenya would be assigned to the domestic and foreign double counted terms (depending on where the value of the intermediate at that stage originated). 

\begin{figure}[h!]
\centering
\caption{\label{fig:KWW}\textsc{KWW Decomposition of Gross Exports}}
\includegraphics[width=1\textwidth, trim= {0 0 0 0}, clip]{"../Figures/KWW".PNG} %trim={<left> <lower> <right> <upper>}
\end{figure}
\FloatBarrier

\subsubsection{Aggregate KWW Decomposition}
The KWW decomposition of gross exports is computed for each of the EAC members and shown in Figure \ref{fig:KWW_fill_ts}. To connect this decomposition to the aggregate measures of GVC integration VS and VS1 obtained from the Leontief decomposition and shown in Figure \ref{fig:VSag_ts}: VS the share of FVA in gross exports is the sum of FVA$_{FIN}$ (7), FVA$_{INT}$ (8) and FDC (9), while VS1 is the sum of DVA$_{INTrex}$ (3), RDV$_{FIN}$ (4), RDV$_{INT}$ (5) and DDC (6) (approximately equal to DVA$_{INTrex}$ here since RDV$_{FIN}$, RDV$_{INT}$  and DDC are close to 0 in all EAC countries).  

\begin{figure}[h!]
\centering
\caption{\label{fig:KWW_fill_ts}\textsc{KWW Decomposition of Gross Exports}}
\includegraphics[width=1\textwidth, trim= {0 0 0 0}, clip]{"../Figures/KWW_fill_ts".pdf} %trim={<left> <lower> <right> <upper>}
\end{figure}
\FloatBarrier

Figure \ref{fig:KWW_fill_ts} shows that double counted items constitute up to 10\% of gross exports in EAC countries, but that most of this double counting occurs with VA produced abroad. The domestic content in intermediate exports that finally returns home is practially 0 for all EAC members, indicating an overall insignificant role of these countries as suppliers of inputs to their own final imports\footnote{This is prevalent in the export composition of high-income countries, see e.g. \citep{Kummritz20162}.}. In all members furthermore the largest share of exports is domestic VA in intermediate exports absorbed by direct importers. Only a small share of DVA in intermediate exports is re-exported indicating that EAC countries predominantly export basic inputs to products manufactured for home consumption in the importing countries. In Uganda furthermore around 40\% of exports constitute DVA in final goods exports. Apart from Tanzania, the share of DVA in final goods exports has increased over time. Tanzania is also the only country with a significant share of FVA in final goods exports, indicating some assembly and processing tasks. Rwanda on the other hand has a high share of FVA in intermediate exports\footnote{Primary the recycling sector as shown in Figure \ref{fig:KWW_fill_sec}.}. \newline  % suggesting moderate upstream integration in GVCs in some industries

\subsubsection{Upstreamness and Downstreamness in GVC Participation}

%\todo[inline]{Make Separate charts for Final and Intermediate Exports: High FVA in final exports indicates downstreamness, while high DVA in intermediate exports indicates upstreamness. By tracking these two variables over time we can see which countries have succeeded in moving up the value chain. See \citep{Kummritz20162}}

The KWW decomposition also lets us asses the position of countries in GVCs regardless of their overall level of GVC integration\footnote{As measured by VS and VS1.}. According to \citet{Kummritz20162} and \citet{wang2013quantifying}, High FVA in final exports relative to total foreign content in exports indicates downstreamness (assembly tasks), while high DVA in intermediate exports relative to total DVA in exports indicates upstreamness (specialization in tasks adding a lot of value to an unfinished product). I follow these authors in computing the ratios as shown in Equations \ref{eq:US} and \ref{eq:DS}, and plot them in Figure \ref{fig:UP_DOWN_ag_ts}.  
\begin{equation} \label{eq:US}
\text{Upstreamness}\quad =\quad \frac{\text{DVA}_{INT} + \text{DVA}_{INTrex} + \text{DDC}}{\text{DVA}_{FIN} + \text{DVA}_{INT} + \text{DVA}_{INTrex} + \text{RDV}_{FIN} + \text{RDV}_{INT} + \text{DDC}}
\end{equation}
\begin{equation} \label{eq:DS}
\text{Downstreamness}\quad =\quad \frac{\text{FVA}_{FIN}}{\text{FVA}_{FIN} + \text{FVA}_{INT} + \text{FDC}},
\end{equation}
%\todo[inline]{Ratios are unintuitive, why not FVA\_FIN over total FIN exports (domestic and foreign)? Google more about upstreamness and downstreamness.}.
% \todo[inline]{also repeat VS and VS1 movements indicating overall GVC integration.}
\begin{figure}[h!] \vspace{-0.3cm}
\centering
\caption{\label{fig:UP_DOWN_ag_ts}\textsc{Upstreamness and Downstreamness Ratios}}
\includegraphics[width=1\textwidth, trim= {0 0 0 0}, clip]{"../Figures/UP_DOWN_ag_ts".pdf} %trim={<left> <lower> <right> <upper>}
% \vspace{-1cm}
\end{figure} 
\FloatBarrier 

From Figure \ref{fig:UP_DOWN_ag_ts} it appears that the smaller countries are situated more upstream in GVCs, but this is also a consequence of them generally exporting less final goods. More meaningful than the levels of these ratios is their change over time. Figure \ref{fig:UP_DOWN_ag_ts} suggests that apart from Tanzania all EAC members became more downstream in their GVC integration, with less domestic content in intermediate exports and more FVA in final goods exports. To better exhibit these findings, Figure \ref{fig:UP_DOWN_ag_growth} shows the difference in the Upstreamness and Downstreamness ratios between 2005 and 2015. It is evident that Uganda and Kenya moved downstream in these years, with more foreign value added going into final goods than intermediate goods, and also more DVA going into final goods. This could, especially for the smaller countries Rwanda and Burundi, also indicate a general increase in Final goods exports that has nothing to do with shifting patterns of GVC integration. %\todo{right?}. 
Only Tanzania appears to have mived slightly upstream over this time period, with more domestic and foreign VA going into intermediate exports. 

\begin{figure}[h!]
\centering
\caption{\label{fig:UP_DOWN_ag_growth}\textsc{Upstreamness and Downstreamness Ratios, Difference 2005-2015}}
\includegraphics[width=1\textwidth, trim= {0 0 0 0}, clip]{"../Figures/UP_DOWN_ag_growth".pdf} %trim={<left> <lower> <right> <upper>}
\end{figure}
\FloatBarrier


\subsection{New Revealed Comparative Advantage}
A popular measure to empirically measure Ricardo's concept of comparative advantage in international trade is the measure of revealed comparative advantage proposed by \citet{balassa1965trade}. It is computed as the share of a sector in gross country exports, divided by the share that of that sector in gross world exports. A ratio above 1 indicates a comparative advantage of the country in this sector. The traditional index based on gross flows however does not take account of double counting in gross exports, and may thus be noisy and misleading. \citet{koopman2014tracing} therefore propose a new index based on VA flows, which considers the domestic VA in gross exports (or domestic GDP in exports, the sum of terms 1-5 of the KWW decomposition). \newline

Figure \ref{fig:NRCA} shows the new revealed comparative advantage (NRCA) for EAC countries in the year 2015. It is evident that all EAC members have a NRCA in agriculture and fishing, which is higher than 10 for agriculture in Uganda, Tanzania and Kenya. Also all EAC members have a comparative disadvantage in core manufacturing sectors such a petro-chemicals, metral products and electrical machinery. The remaining sectors show more heterogeneity across EAC countries, where Kenya appears to be different from the other countries. In Uganda, Tanzania, Rwanda, Burundi and South Sudan activities of private households (self-employment) seem to have a strong comparative advantage, and also maintenance and repair activities have a strong comparative advantage, whereas in Kenya both sectors appear to have a comparative disadvantage. \newline

\begin{figure}[h!]
\centering
\caption{\label{fig:NRCA}\textsc{New Revealed Comparative Advantage in 2015}}
\includegraphics[width=1\textwidth, trim= {0 0 0 0}, clip]{"../Figures/NRCA".pdf} %trim={<left> <lower> <right> <upper>}
\end{figure}
\FloatBarrier

As we have 10 years of data, it will be interesting to see whether there were significant changes in the revealed comparative advantage of sectors. Figure \ref{fig:NRCA_growth} therefore shows the annualized growth rate in NRCA over the 2005-2015 period. It is evident that comparative advantage has not changed much in agriculture, with minor annual gains or losses within the [-2\%, 2\%] range. All EAC members seem to have gained comparative advantage in fishing, particularly Uganda and Tanzania with gains of 2.2\% and 5.3\%, respectively. Also all EAC members had lost comparative advantage in mining, especially Uganda. All EAC countries have gained comparative advantage in re-exporting goods. In other sectors developments are rather heterogeneous. Uganda for example appears to have gained comparative advantage in exporting transport equipment by around 5.1\% annually, whereas Rwanda lost comparative advantage in the same sector by -4.1\% annually. There were also hardly any movements of sectors from comparative disadvantage (NRCA $< 1$) to comparative advantage (NRCA $> 1$) in this time period, only Tanzania's mining sector moved to a slight comparative disadvantage of 0.99 in 2015, from 1.23 in 2005. 

%It is evident from the many straight lines in the graph that comparative advantage did not change much over this time period. In Uganda, Rwanda, Burundi and South Sudan there was no transition of a comparative disadvantage sector (NRCA < 1) to a comparative advantage sector (NRCA > 1) over this period. The gratest fluctuations are exhibited by Tanzania where the data suggest that NRCA in Agriculture and Fishing has increased from around 10 to around 13. 

% \todo[inline]{Replace Figure with a growth rates bar chart just like the figure above. Or better: Try the dot plot straight}

\begin{figure}[h!]
\centering
\caption{\label{fig:NRCA_growth}\textsc{NRCA Annualized 2005-2015 Growth Rate}}
\includegraphics[width=1\textwidth, trim= {0 0 0 0}, clip]{"../Figures/NRCA_growth".pdf} %trim={<left> <lower> <right> <upper>}
\end{figure}
\FloatBarrier

% In the other countries movements are very gradual, in Uganda comparative advantage in Agriculture has increased very slightly from 17.8 in 2005 to 18 in 2015, whereas in Kenya it has dropped from 15.2 to 13.3 over the same period. 


\subsubsection{NRCA Relative to the EAC}
Figures \ref{fig:NRCA} and \ref{fig:NRCA_growth} have shown that relative to the rest of the world, EAC members exhibit similar patterns of comparative advantage with a general advantage in agriculture and disadvantage in manufacturing. This could be constitutive to forming a common trade block with the rest of the world, supported by a currency union as is currently planned for 2024/25. Nevertheless comparing the EAC with the rest of the world may mask rivalries and shifts in comparative advantage between member countries. To uncover these dynamics, comparative advantage is also computed relative to the EAC, as the share of a sector in country VA exports to the share of the sector in EAC VA exports. %\todo{Which also includes EAC countries, should it?}. 
Figure \ref{fig:NRCA_EAC} shows the results, where in addition to the bars for the year 2015, I have added a blue dot giving the 2005 value - to avoid having to draw two charts while still enabling a comparison over time as well. \newline 

It is evident from Figure \ref{fig:NRCA_EAC} that, relative to other EAC members, Uganda has a comparative advantage in agriculture, Tanzania has a comparative advantage in fishing, Rwanda has a comparative advantage in mining, and Kenya has a comparative advantage in core manufacturing sectors such as wood and paper, petro-chemicals, metal products and electrical machinery. In addition, it appears that Rwanda and Burundi, and to a weaker extent Uganda, have a comparative advantage in construction, maintenance and repairs, wholesale and retail trade, whereas Tanzania appears to have a comparative advantage in other manufacturing, recycling, and financial and business services. Other sectors are more heterogeneous between countries and sectors. The aforementioned comparative advantage relations appear to be stable since 2005, without major shifts within or across sectors between 2005 and 2015. 

\begin{figure}[h!]
\centering
\caption{\label{fig:NRCA_EAC}\textsc{NRCA Relative to EAC}}
\includegraphics[width=1\textwidth, trim= {0 0 0 0}, clip]{"../Figures/NRCA_EAC".pdf} %trim={<left> <lower> <right> <upper>}
\end{figure}
\FloatBarrier



\section{GVCs and Industrial Development}
Several papers have been written focussing on the global links between GVC integration and industrial development. As one of the first, \citet{Kummritz20161} assessed the role of GVCs for labour productivity and domestic value added using OECD ICIO's for 61 countries and 34 industries from 1995-2011. He achieves identification using a novel IV strategy where a value added trade resistance index combining third country trade costs with industry-specific technological variables induces exogenous variation in GVC participation. He shows that an increase in GVC participation leads to higher domestic value added and productivity for all countries independent of their income levels. His results imply that a 1 percent increase in backward GVC participation (I2E) leads to 0.11\% higher domestic value added in the average industry, and  a 1 percent increase in forward GVC participation (E2R) leads to 0.60\% higher domestic value added and to 0.33\% higher labour productivity \citep{Kummritz20161}. \newline


The literature discussed by \citet{Kummritz20161} outlines several channels through which GVC participation (in north-south value chains) increases the value added and productivity of its participants. The main channels are learning-by-doing, technology transfer or spillovers, gains from specialization in comparative advantage tasks, and terms of trade effects  \citep{Kummritz20161}. \newline

\citet{piermartini2014knowledge} for example use industry-level R\&D and patent data for a sample of 29 countries during the period 2000-2008, and show that knowledge spillovers increase with the intensity of supply chains linkages between countries, and that these spillovers are larger in magnitude than spillovers from traditional trade flows. Similar evidence is presented by \citet{benz2015trade}, which use firm-level data to show that offshoring leads to knowledge spillovers, and further that forward spillovers (from producers to users if intermediate inputs) are stronger than backward spillovers. \newline

\citet{Kummritz20161} notes that GVCs don't necessarily need to benefit developing countries as there could be adverse terms of trade effects or decreases in productive endowments from heavy engagement in them (which is also shown in some theoretical models such as \citet{baldwin2014trade}). An argument made by \citet{kummritz2015global} is also that GVCs might substitute foreign for domestic suppliers, but his own empirical research suggests that 
foreign value added works as a complement rather than a substitute to domestic value added, and that GVC participation benefits the domestic economy along the value chain if certain prerequisites are met.  \citet{kummritz2015global} finds no significant effect of GVC participation on low-income countries, which he attributes to the low absorptive capacity to benefit from technology spillovers. The micro-papers discussed however point towards an overwhelmingly positive effect. \newline

% He concludes that whether GVC integration provides net benefits for a particular country or sector is theoretically ambiguous. \newline


% \citet{foster2015global} discusses industrial development in Africa through GVCs in the context of upgrading, 

% Four types of upgrading are often distinguished (Humphrey, 2004), the four types being: (i) process upgrading, which involves increased productivity in existing activities within a GVC; (ii) product upgrading, which is the movement into higher value-added products within a GVC; (iii) functional upgrading, which involves the movement into more technologically sophisticated or more integrated aspects of a production process; and (iv) inter-sectoral or chain upgrading, which involves a movement into higher value-added supply chains. The.

% In their analysis \citet{foster2015global} construct three alternative indicators, intended to capture one or more aspects of upgrading within GVCs. 
% \todo[inline]{-> They compare export unit values to export market shares The use trade data aggregated to EORA Sectors. }. 

% \todo[inline]{Compute Herfindahl Index of Value added exports and see how it varies with GVC involvement (Sum of VS and VS1 like in GVC in Africa Paper). Also compute oher upgrading and export difersification indicators. Take total GVC Involvement measure as in paper VS + VS1.}

% \todo[inline]{Combine GVC with your ES Analysis and look at innovtion measures.}

Further empirical evidence on the relationship between domestic value chains i.e. the fragmentation of domestic production and GVCs is provided by \citet{beverelli2019domestic}. They find that, across countries at different stages of development, higher domestic integration by 1 standard deviation raises subsequent GVC integration through backward linkages (I2E) by 0.4\%. They also find that domestic value chain integration explains up to 30\% of overall GVC integration. They explain these results with fixed costs of fragementation and switching suppliers: "high fragmentation costs allow, due to their sunk nature, DVCs to act as stepping stones to GVCs" \citep{beverelli2019domestic}. The results also imply that improving domestic economic integration would further GVC integration in the medium run. 

\subsection{GVC Integration and  Growth}

A natural idea to assess the impact of GVC integraion on growth is to investigate if higher imported or re-exported content in exports is associated with higher domestic value added produced i.e. higher GDP. This can be examined, following papers such as \citet{kummritz2015global}, using a simple specification regressing the log of VA on the imported content share (I2E) and the re-exported content share (E2R). The specification used is

\begin{equation} \label{eq:GROWTH_HDFE}
log(VA_{cst}) = \sum_{i=0}^p \beta_{1i} I2E_{cs,t-i} + \sum_{i = 0}^p \beta_{2i} E2R_{cs,t-i}  + \alpha_{cs} + \beta_{ct} +\gamma_{st} + \epsilon_{cst},
\end{equation}

where $p$ is a suitable number of lags to include in the regression to allow changes in production (in particular I2E) to feed into greater productivity with a lag. The specification including lags is applied by \citet{kummritz2015global} and theoretically justified by the dynamic GVC model of \citet{LiLiu2015moving} where the effect of GVC participation on domestic VA accrues in the next period. In the most general setting we can control for 3 sets of unobservable effects: country-sector effects ($\alpha_{cs}$), country-year effects ($\beta_{ct}$) and sector-year effects ($\gamma_{st}$). A similar specification, without dynamic effects but with an instrument for I2E and E2R, is estimated by \citet{Kummritz20161} who finds a positive effect of both GVC measures, and further that OLS and IV give similar results. This gives me some confidence that constructing an instrumental variable for GVC integration (which is difficult for these countries and the EORA data), might not add much to the estimation. \citet{kummritz2015global} also notes that in the absence of an instrument, fixed effects and lags are the best specification choices towards a careful causal interpretation of the results.  \newline

 There is also the possibility to estimate a first-difference estimator which is more efficient if the autocorrelation in the error term is $> 0.5$\footnote{In particular, if $\epsilon_{cst} = \rho \epsilon_{cs,t-1} + u_{cst}$, then $var(\epsilon_{cst}) = \rho^2 \sigma^2_\epsilon + \sigma^2_u$ but for the first-differenced model $var(\Delta \epsilon_{cst}) = var(\epsilon_{cst} - \epsilon_{cs,t-1}) = var(\rho \epsilon_{cs,t-1} + u_{cst} - \epsilon_{cs,t-1}) = (\rho-1)^2 \sigma^2_\epsilon + \sigma^2_u$. So the FD estimator is more efficient if $(\rho-1)^2<\rho^2$ or if $\rho > 0.5$. } 

\begin{equation} \label{eq:GROWTH_HDFE}
\Delta log(VA_{cst}) = \sum_{i=0}^p \beta_{1i} \Delta I2E_{cs,t-i} + \sum_{i = 0}^p \beta_{2i} \Delta E2R_{cs,t-i}  + \Delta\beta_{ct} + \Delta\gamma_{st} + \Delta\epsilon_{cst}.
\end{equation}

For this regression I use data for Uganda, Tanzania, Kenya, Rwanda and Burundi. South Sudan was excluded due to unreliable data. From the data for these countries I further remove sectors where I2E or E2R are greater than 1 or smaller than 0. This should usually not be the case, but is the case in recycling and re-import/export sectors in Rwanda, Tanzania and Burundi, in the financial intermediation and business sectors in Burundi, Rwanda and Uganda, in Kenyan and Ugandan electricity gas and water, and Kenyan private households and other sectors. This is likely due to both bad data quality and very unusual economic activity in these sectors. For the estimation these six sectors (REC, REI, FIB, EGW, PHH and OTH in Table \ref{tab:sec}) are removed from the sample. %Other sectors which have a too high re-export ratio and are removed from the sample are are Kenyan private housholds and  Kenyan others. 
Summary statistics for the excluded sectors are shown in Table \ref{tab:EXCL_SEC}.

% Table created by stargazer v.5.2.2 by Marek Hlavac, Harvard University. E-mail: hlavac at fas.harvard.edu
% Date and time: Mon, May 03, 2021 - 12:57:45 PM
\begin{table}[h!] \centering 
  \caption{\label{tab:EXCL_SEC}\textsc{Excluded Sectors}}
  \vspace{2mm}
\begin{tabular}{ llrrrrr} \toprule
Sector & GVC Measure  & N & Mean & SD & Min & Max \\ 
\midrule
EGW & I2E & $55$ & $0.143$ & $0.055$ & $0.063$ & $0.248$ \\ 
EGW & E2R & $55$ & $0.750$ & $0.539$ & $0.158$ & $1.828$ \\ 
FIB & I2E & $55$ & $0.059$ & $0.024$ & $0.027$ & $0.110$ \\ 
FIB & E2R & $55$ & $4.603$ & $5.225$ & $0.294$ & $19.275$ \\ 
OTH & I2E & $55$ & $0.224$ & $0.122$ & $0.077$ & $0.493$ \\ 
OTH & E2R & $55$ & $3.164$ & $6.194$ & $0.072$ & $21.512$ \\ 
PHH & I2E & $55$ & $0.307$ & $0.179$ & $0.077$ & $0.658$ \\ 
PHH & E2R & $55$ & $3.071$ & $6.240$ & -$0.060$ & $21.512$ \\ 
REC & I2E & $55$ & $0.435$ & $0.258$ & $0.173$ & $1.018$ \\ 
REC & E2R & $55$ & $0.027$ & $0.058$ & -$0.108$ & $0.137$ \\ 
REI & I2E & $55$ & $0.821$ & $0.384$ & $0.352$ & $1.787$ \\ 
REI & E2R & $55$ & $0.028$ & $0.100$ & -$0.223$ & $0.138$ \\ \bottomrule
\end{tabular} 
\end{table} 

I end up with a balanced panel of $N = 1100$ observations in $CS = 100$ country-sectors (5 countries, 20 sectors) and $T = 11$ time periods. Summary statistics for overall variation as well as between and within country-sectors are shown in Table \ref{tab:SUMM_GROWTH}. %, where I also added a the domestic content in exports computed as $DVA_{EX} = EX \times (1 - I2E)$ where $EX$ is a country-sectors gross exports. 

% Table created by stargazer v.5.2.2 by Marek Hlavac, Harvard University. E-mail: hlavac at fas.harvard.edu
% Date and time: Mon, May 03, 2021 - 2:18:27 PM
\begin{table}[h!] \centering 
  \caption{\label{tab:SUMM_GROWTH}\textsc{Summary Statistics of Variables}}
%  \vspace{2mm}
  \begin{center}
\begin{tabular}{ llrrrrr} \toprule
Variable & Trans. & N/T & Mean & SD & Min & Max \\ \midrule
VA & Overall & $1,100$ & $499,353$ & $993,836$ & -$1,063$ & $11,335,675$ \\ 
VA & Between & $100$ & $499,353$ & $952,846$ & $3,247$ & $7,854,686$ \\ 
VA & Within & $11$ & $499,353$ & $296,745$ & -$3,084,848$ & $3,980,341$ \\ 
%DVA$_{EX}$ & Overall & $1,100$ & $64,055$ & $176,676$ & $382$ & $1,727,299$ \\ 
%DVA$_{EX}$ & Between & $100$ & $64,055$ & $174,711$ & $627$ & $1,474,635$ \\ 
%DVA$_{EX}$ & Within & $11$ & $64,055$ & $31,114$ & -$412,498$ & $316,719$ \\ 
I2E & Overall & $1,100$ & $0.20$ & $0.13$ & $0.03$ & $0.70$ \\ 
I2E & Between & $100$ & $0.20$ & $0.12$ & $0.04$ & $0.59$ \\ 
I2E & Within & $11$ & $0.20$ & $0.04$ & $0.01$ & $0.36$ \\ 
E2R & Overall & $1,100$ & $0.15$ & $0.10$ & -$0.05$ & $0.62$ \\ 
E2R & Between & $100$ & $0.15$ & $0.09$ & $0.01$ & $0.51$ \\ 
E2R & Within & $11$ & $0.15$ & $0.02$ & -$0.00$ & $0.30$ \\ \bottomrule
\end{tabular} 
 \end{center}
% \footnotesize{\emph{Note:} $I2E_P = I2E \times 100$, $I2E_P = I2E \times 100$}
\end{table} 
\FloatBarrier 

To further expose the manufacturing sectors in these countries, whose productivity is likely most affected by changing integration in GVCs, I also run regressions for a sub-sample of manufacturing sectors (FBE, TEX, WAP, PCM, MPR, ELM, TEQ, MAN in Table \ref{tab:sec}). To better understand the data and the manufacturing sub-sample of sectors, Figure \ref{fig:GROWTH_REG_TS} visualizes the data, where each orange line is a manufacturing sector in some EAC country, and grey lines are other sectors. The red line represents the median value across all manufacturing sectors in a given year and the black line is the median across all sectors.  


\begin{figure}[h!]
\centering
\caption{\label{fig:GROWTH_REG_TS}\textsc{Time Series of Variables}}
\includegraphics[width=1\textwidth, trim= {0 0 0 0}, clip]{"../Figures/GROWTH_REG_TS".pdf} %trim={<left> <lower> <right> <upper>}
\end{figure}
\FloatBarrier

It is evident from Figure \ref{fig:GROWTH_REG_TS} that manufacturing sectors have a lower than average VA, a higher I2E share, and a lower E2R share, indicating that these sectors import more inputs that other economic sectors but export more final goods. The trend line is broadly parallel to the overall trend, but in the VA chart it appears that manufacturing sectors grew slower than the average. For a final step of visual investigation, \ref{fig:GROWTH_REG_Hists} shows histograms conveying the same information as Figure \ref{fig:GROWTH_REG_TS}. They show that in terms of VA manufacturing is a bit lower. % but well-centered in the distribution. 
In terms of I2E, manufacturing sectors make up the bulk of the upper part of the distrubution, whereas for E2R the opposite is the case. In both cases the distribution is a bit skewed to the right. 

\begin{figure}[h!]
\centering
\caption{\label{fig:GROWTH_REG_Hists}\textsc{Histograms of Variables}}
\includegraphics[width=1\textwidth, trim= {0 0 0 0}, clip]{"../Figures/GROWTH_REG_Hists".pdf} %trim={<left> <lower> <right> <upper>}
\end{figure}
\FloatBarrier

Regarding the specification in Eq. \ref{eq:GROWTH_HDFE}, I first select the appropriate lag length by running the regression with fixed effects and first-differences and examining up to which order lags of I2E and E2R affect value added. Together with some judgement I opt for $p = 2$, which is a sensible choice as it might take up to 2 years for an innovation or change in supply chain to fully dissipate to output and productivity. Then, to determine whether the specification given in Eq. \ref{eq:GROWTH_HDFE} is appropriate, I run a series of Hausman tests, including 2 lags of I2E and E2R in the regression. The first test evaluates the consistency of the random effects estimator against the the  simple fixed effects estimator with country-sector fixed effects, using the original $\chi^2$ distributed quadratic form proposed by \citet{hausman1978specification}. It rejects the null of random effects consistency $\chi^2_6 = 73.05$, $P < 0.01$. Then, I demean the data by country-sector and run a second Hausman test with country-year fixed effects. This test also rejects $\chi^2_6 = 53.4$, $P < 0.01$. Finally, I iteratively demean the data by country-sector and country-year until convergence and run a third Hausman test against sector-year fixed effects. This test also rejects $\chi^2_6 = 48.55$, $P < 0.01$, but a robust version of the test based on an auxiliary regression as specified in \citet{wooldridge2010econometric}\footnote{This test is based on an auxiliary specification $\tilde{y}_{it} = \tilde{X}_{it}\beta + \dot{X}_{it}\gamma + \epsilon_{it}$ that can be estimated with robust standard errors, where  $\tilde{y}_{it} $ and $\tilde{X}_{it}\beta$ are the quasi-demeaned data for RE estimation and $\dot{X}_{it}$ are the time-demeaned predictors capturing the individual-variation in $X$. The test is an F-test of the exclusion restriction of $\dot{X}_{it}$. If the test rejects, RE is likely inconsistent. See \citet{wooldridge2010econometric} sec. 10.7.3.} fails to reject the null $\chi^2_6 = 9.37$, $P = 0.15$. I nevertheless keep the sector-year fixed effects in the model, as the coefficient is practically identical to the one without them, and keeping them reduces a bit the serial correlation in the error term. \newline

Serial correlation in the error term, $\epsilon_{cst} = \rho \epsilon_{cs,t-1} + u_{cst}$ for $\rho > 0$ might obscure conducting inference on the model and make the first-difference estimator more efficient. In practice it is difficult to determine the value of $\rho$, given that the are errors in the fixed effects model are unobservable\footnote{$\hat{\epsilon}_{cst}$ is not a clean estimate of $\epsilon_{cst}$, but an estimate of the multiply-centered version of $\epsilon_{cst}$.}, but I use $\hat{\epsilon}_{cst}$ to obtain a crude estimate using OLS. After estimating Eq. \ref{eq:GROWTH_HDFE} with the full set of fixed effects, I estimate $\hat{\rho}_{FE} = 0.53$, $P<0.01$\footnote{\citet{wooldridge2010econometric} sec. 10.5.4 observes, under the null of no serial correlation in the errors, the residuals of a FE model must be negatively serially correlated, with $cor(\hat{u}_{it}, \hat{u}_{is})=-1/(T-1) = -0.1$ with $T = 11$ in this case.}. A formal panel-test based on the residuals of the first-differenced model also rejects the null of no serial correlation in the error term $\hat{\rho}_{FD} = -0.011$, $P=0.77$ and $P[\hat{\rho}_{FD} \neq -0.5]=<0.01$\footnote{This is the case because, for each $t > 1$, $var(\Delta u_{it}) = var(u_{it} - u_{i,t-1}) = var(u_{it}) + var(u_{i,t-1}) = 2\sigma^2$ with the assumptions of no serial correlation in $u_t$ and constant variance. Because the residual has a zero mean and symmetric ACF, the covariance is $E[\Delta u_{it}\Delta u_{i,t+1}] = E[(u_{it} - u_{i,t-1})(u_{i,t+1} - u_{it})] = E[u_{it} u_{i,t+1}] - E[u_{it}^2] - E[u_{i,t-1} u_{i,t+1}] + E[u_{i,t-1} u_{it}] = -E[u_{it}^2] = -\sigma^2$, because of the no serial correlation assumption. Because the variance is constant across t, $cor(\Delta u_{it},  \Delta u_{i,t-1}) = cov(\Delta u_{it},  \Delta u_{i,t+1})/var(\Delta u_{it}) = -\sigma^2/2\sigma^2 = -0.5$.}. First-differencing in itself does not remove terms $\Delta\beta_{ct}$ and $\Delta\gamma_{st}$, corresponding to unobserved country-year or sector-year specific shocks from the equation, which may also be correlated with the explanatory variables and bias the coefficient estimates in the FD-equation. Running Hausman tests for the presence of these effects in the first-difference equation yields inconclusive results, the outcome depends on the method used to run the Hausman test. There is also a danger that putting fixed effects in a first-differenced equation estimated on data of not very high quality removes too much useful information and aggravates the impact of measurement error in the data, yielding attenuation bias.   \newline

The approach I will adopt, is to estimate 3 models: The simple FD specification, the FD specification with country-year and sector-year FE, and a FE specification with the full set of fixed effects. \newline 

These models are first estimated in log-level form as specified in Eq. \ref{eq:GROWTH_HDFE}, which gives the percentage change in VA in response to a 0.01 unit increase in the GVC indicators (I2E and E2R), expressed as a share of gross exports. But as Figure \ref{fig:GROWTH_REG_Hists} shows, different sectors have vastly different I2E and E2R shares, thus for a sector with a low level of GVC integration, an increase in the share of 0.01 may imply a vastly greater degree of restructuring of production and potential gains or losses than a 0.01 increase for a sector already quite integrated in GVCs. To take account of these different levels of GVC integration, I run an additional set of 3 regressions where the log of the share instead of the share is included on the right-hand-side, thus giving an elasticity as the percentage change in VA from a percentage change in the share. Finally, since \citet{Kummritz20161} uses the log of the values of I2E and E2R (not expressed as a percentage of exports but as VA content in exports), and taking the log of a large value may seem more natural than taking the log of a share, I also estimate 3 regressions with this classical elasticity specification. Thus in total I report 9 regressions:  3 different estimators times 3 different transformations of the independent variables. Table \ref{tab:VAGRREG} reports the results\footnote{Estimation was done using the R package \textit{fixest} \citep{fixest2018}.}. \newline %, including the FE model without sector-year fixed effects. 
%  To the right of these specifications, Table  \ref{tab:VAGRREG} also shows log-log / elasticity estimates where following \citet{Kummritz20161} I use the log of the value of imported content (VS) and re-exported content (VS1) in exports, not the the log of their export share, as right-hand-side variables. \newline 

The coefficients from all specifications show a negative contemporaneous relationship between VA and I2E. This is probably a quite mechanical result by the nature of the close relation between VA and I2E being the foreign content in production and exports. A domestic shock of any form may cause VA to increase/decrease and the imported share to fall/rise in the current period. It is therefore more interesting to examine the lagged relationship of I2E and VA. Here the coefficients of the preferred FD specification signify a significant positive effect. In the first regression on the shares (S), only the second lag of I2E is significant at the 10\% level, implying that a 0.01 unit increase in I2E is associated with a 0.31\% increase in VA after two years. When using the log of the share (ES) however the coefficients on both lags are positive and significant, with a 1\% increase in the I2E share associated with a  0.29\% increase in VA after two years. The specification using the log of foreign VA in exports (E) yields that a 1\% increase yields a 0.14\% increase in VA after two years. The FD-TFE and FE specifications do not pick up an effect after one year but a larger effect after 2 years. As noted, because of significant serial correlation, FD is more efficient here\footnote{I have argued before for the likely consistency of all estimators on the basis of inconclusive Hausman Tests for the inclusion of fixed effects in the first difference equation.}.    % this discrepancy between FD and FE indicates that the findings are not very robust. 


\begin{table}[h!]
\centering
\caption{\label{tab:VAGRREG} \textsc{Value Added Regressions}}
\resizebox{1\textwidth}{!}{
\begin{tabular}[t]{lccccccccc} \toprule
% Dependent Variable:&\multicolumn{6}{c}{log\_VA}\\
 \textit{Model:}    & FD S & FD-FE S & FE S & FD ES & FD-FE ES & FE ES & FD E & FD-FE E & FE E\\
                             &(1) & (2) & (3) & (4) & (5) & (6) & (7) & (8) & (9)\\ \midrule 
                 &&&&&&&&&  \\
(Intercept)&0.0716$^{***}$ &    &    & 0.0746$^{***}$ &    &    & 0.0390$^{***}$ &    &   \\
  &(0.0041) &    &    & (0.0029) &    &    & (0.0022) &    &   \\\\
I2E&-1.593$^{***}$ & -5.254$^{***}$ & -5.354$^{***}$ & -0.1323$^{***}$ & -0.7345$^{***}$ & -0.4249$^{***}$ & -0.4931$^{***}$ & -0.5564$^{***}$ & -0.5420$^{***}$\\
  &(0.3944) & (0.8604) & (0.7238) & (0.0474) & (0.1333) & (0.1233) & (0.0403) & (0.0504) & (0.0730)\\\\
L1.I2E&-0.0526 & -0.4503 & -0.8266 & 0.1756$^{***}$ & 0.1046 & 0.0188 & 0.1092$^{***}$ & 0.0208 & 0.0092\\
  &(0.3921) & (0.6753) & (1.379) & (0.0228) & (0.0887) & (0.1296) & (0.0223) & (0.0353) & (0.0488)\\\\
L2.I2E&0.3079$^{*}$ & 0.6766 & 1.307$^{**}$ & 0.1114$^{***}$ & 0.0340 & -0.0127 & 0.0265$^{*}$ & 0.0436$^{*}$ & 0.0630$^{**}$\\
  &(0.1678) & (0.5324) & (0.6186) & (0.0230) & (0.0254) & (0.0521) & (0.0143) & (0.0229) & (0.0241)\\\\
E2R&3.871$^{***}$ & 1.380$^{***}$ & 1.638$^{**}$ & 0.9962$^{***}$ & 0.8492$^{***}$ & 0.8968$^{***}$ & 0.8271$^{***}$ & 0.8950$^{***}$ & 0.8585$^{***}$\\
  &(0.8564) & (0.4795) & (0.6379) & (0.0388) & (0.0633) & (0.0596) & (0.0460) & (0.0242) & (0.0380)\\\\
L1.E2R&1.378$^{***}$ & 0.2976 & -0.2803 & 0.1271$^{***}$ & 0.0349 & 0.0862 & 0.0800$^{***}$ & -0.0073 & 0.0487\\
  &(0.2895) & (0.2985) & (0.4063) & (0.0464) & (0.0323) & (0.0756) & (0.0237) & (0.0266) & (0.0567)\\\\
L2.E2R&0.4809$^{**}$ & -0.1346 & -0.7844$^{**}$ & 0.0954$^{***}$ & -0.0273 & -0.1421$^{**}$ & 0.0244$^{*}$ & 0.0039 & -0.0536\\
  &(0.1968) & (0.1963) & (0.3169) & (0.0262) & (0.0261) & (0.0606) & (0.0136) & (0.0197) & (0.0372)\\\\
\midrule \emph{Fixed-Effects:} &   &   &   &   &   &  \\
cs (N) & -- & -- & 100 & -- & -- & 100 & -- & -- & 100\\
cy (N) & -- & 40 & 45 & -- & 40 & 45 & -- & 40 & 45\\
sy (N) & -- & 160 & 180 & -- & 160 & 180 & -- & 160 & 180\\
\midrule
% \emph{Fit statistics}&  & & & & & \\
Cluster SE & cs & cs cy sy & cs cy sy & cs & cs cy sy & cs cy sy & cs & cs cy sy & cs cy sy\\
Observations & 800&800&900&798&798&898&798&798&898\\
R$^2$ & 0.301&0.802&0.998&0.701&0.961&1.00&0.758&0.970&1.00\\
Within R$^2$ & &0.483&0.467&&0.898&0.881&&0.921&0.900\\ \bottomrule \\[-1em]
\multicolumn{7}{l}{\small \textit{Note:} The dependent variable is the natural log of VA, which is regressed on the  export shares}   & \multicolumn{3}{r}{$^{*}$p$<$0.1; $^{**}$p$<$0.05; $^{***}$p$<$0.01} \\ [-0.2em]
\multicolumn{10}{l}{\small \quad \quad \quad (1)-(3), natural log of the shares (4)-(6), and natural log of values (7)-(9) of I2E and E2R.} \\
%\multicolumn{7}{l}{\textsuperscript{} * p < 0.1, ** p < 0.05, *** p < 0.01}\\
%\multicolumn{7}{l}{\emph{Signif. Codes: ***: 0.01, **: 0.05, *: 0.1}}\\
\end{tabular}
}
\end{table}
\FloatBarrier




For E2R, the results imply a large contemporaneous relationship with VA, with an elasticity around 1. Also in this case caution needs to be exerted towards interpreting this as a structural shift in production. It could be for example that supply chain shocks contemporaneously lead all participating countries to export more and thus trigger an increase in both VA and E2R. However, the contemporaneous relationship between VA and the E2R is less obvious than the relationship between VA and I2E. The lagged coefficients on E2R for the FD S equation also show a large impact, implying that that a 0.01 unit increase in E2R yields a 1.86\% gain in GDP growth over two years. Curiously the elasticity specifications show a much weaker impact, with the elasticity of the share (FD-ES) implying that a 1\% increase in E2R gives a 0.22\% growth gain within two years, and the classical elasticity (FD E) implying that a 1\% increase in the foreign VA in exports yields a 0.1\% increase in growth within two years. The FD-FE and FE specifications show mostly insignificant results on the lags, with the exception of two significant and negative effects for the FE S and FE ES specifications. These could also be the result of some attenuation bias; the quality of this data after removing all cross-country, sectoral and temporal variation between countries and sectors is likely not very high.  I note that interpreting only the coefficients on the lags as causal, \textit{ceteris paribus} any contemporaneous causal effect, implies that the combined effect of the lags constitutes a lower bound estimate of the true effect of GVC integration on growth.  % \newline

It should also be noted here that there are some significant outliers affecting especially the log-level regressions. The omission of these does not dramatically change the conclusion of the analysis but can shift the relative magnitude of coefficients on the lagged values a bit. For example excluding the 6 most influential data points in the first FD specification yields coefficients $0.25$ and $0.43$ on L1.I2E and L2.I2E. \newline 

To examine the effects of outliers on the coefficients in a more comprehensive manor, the regressions are re-estimated using a robust MM estimation method following \citet{yohai1987high} and \citet{koller2011sharpening}, that downweights outliers and high-leverage data points using a highly efficient Iteratively Reweighted Least Squares (IRLS) procedure with 50\% breakdown point and 95\% asymptotic efficiency for normal errors\footnote{Estimation is done by a robust MM proceudure using IRLS with a bi-square redescending score function, resulting in a highly robust and highly efficient estimator (with 50\% breakdown point and 95\% asymptotic efficiency for normal errors), Implemented in the R package \textit{robustbase} \citep{rousseeuw2009robustbase}.}. The robust coefficient estimates are reported in Table \ref{tab:VAGRREG_R}. In preparation for estimation, the input data have been suitably differenced / demeaned to yield the same least squares estimates as reported in Table \ref{tab:VAGRREG}\footnote{Multi-dimensional centering and differencing in preparation for robust estimation was done using the R package \textit{collapse} \citep{collapse2021}.}. 

% Table created by stargazer v.5.2.2 by Marek Hlavac, Harvard University. E-mail: hlavac at fas.harvard.edu
% Date and time: Sat, May 08, 2021 - 12:40:45 AM
\begin{table}[!htbp] \centering 
\caption{\label{tab:VAGRREG_R} \textsc{Value Added Regressions: Robust MM Estimates}}
\resizebox{\textwidth}{!}{
\begin{tabular}[t]{lccccccccc} \toprule
% Dependent Variable:&\multicolumn{6}{c}{log\_VA}\\
 \textit{Model:}    & FD S & FD-FE S & FE S & FD ES & FD-FE ES & FE ES & FD E & FD-FE E & FE E\\
                             &(1) & (2) & (3) & (4) & (5) & (6) & (7) & (8) & (9)\\ \midrule 
                 &&&&&&&&&  \\
(Intercept) & 0.0722$^{***}$ & -0.00002 & -0.0012 & 0.0712$^{***}$ & 0.0006 & -0.0001 & 0.0305$^{***}$ & 0.0006 & -0.0001 \\ 
  & (0.0031) & (0.0011) & (0.0012) & (0.0026) & (0.0007) & (0.0010) & (0.0022) & (0.0006) & (0.0008) \\ 
  & & & & & & & & & \\ 
 I2E & -0.9218$^{***}$ & -3.9477$^{***}$ & -5.3803$^{***}$ & -0.1070$^{***}$ & -0.7019$^{***}$ & -0.4799$^{***}$ & -0.4988$^{***}$ & -0.5316$^{***}$ & -0.4942$^{***}$ \\ 
  & (0.2354) & (0.3435) & (0.2889) & (0.0345) & (0.0604) & (0.0795) & (0.0335) & (0.0232) & (0.0352) \\ 
  & & & & & & & & & \\ 
 L1.I2E & 0.3766$^{**}$ & 0.1692 & 0.6597$^{**}$ & 0.1457$^{***}$ & 0.0883$^{*}$ & -0.0906 & 0.0688$^{***}$ & 0.0214 & -0.0325 \\ 
  & (0.1608) & (0.1661) & (0.2827) & (0.0295) & (0.0533) & (0.0827) & (0.0258) & (0.0196) & (0.0351) \\ 
  & & & & & & & & & \\ 
 L2.I2E & 0.4294$^{***}$ & 0.1286 & 0.7084$^{***}$ & 0.1208$^{***}$ & -0.0046 & 0.0055 & 0.0401$^{***}$ & 0.0309$^{**}$ & 0.0420$^{**}$ \\ 
  & (0.1381) & (0.1772) & (0.2108) & (0.0241) & (0.0251) & (0.0340) & (0.0124) & (0.0145) & (0.0187) \\ 
  & & & & & & & & & \\ 
 E2R & 4.8046$^{***}$ & 0.8824$^{***}$ & 1.1079$^{***}$ & 1.0401$^{***}$ & 0.8666$^{***}$ & 0.9033$^{***}$ & 0.8661$^{***}$ & 0.9034$^{***}$ & 0.8794$^{***}$ \\ 
  & (0.4523) & (0.3124) & (0.2357) & (0.0344) & (0.0224) & (0.0295) & (0.0348) & (0.0098) & (0.0199) \\ 
  & & & & & & & & & \\ 
 L1.E2R & 1.1457$^{***}$ & 0.2295 & -0.0044 & 0.1195$^{**}$ & 0.0047 & 0.0270 & 0.0673$^{**}$ & -0.0060 & 0.0147 \\ 
  & (0.2623) & (0.1544) & (0.2139) & (0.0544) & (0.0162) & (0.0371) & (0.0320) & (0.0138) & (0.0300) \\ 
  & & & & & & & & & \\ 
 L2.E2R & 0.8208$^{***}$ & -0.0220 & -0.4501$^{***}$ & 0.0885$^{***}$ & -0.0223 & -0.0902$^{**}$ & 0.0145 & 0.0040 & -0.0231 \\ 
  & (0.2061) & (0.1328) & (0.1403) & (0.0224) & (0.0177) & (0.0350) & (0.0146) & (0.0104) & (0.0169) \\ 
  & & & & & & & & & \\ 
\midrule \emph{Fixed-Effects:} &   &   &   &   &   &  \\
cs (N) & -- & -- & 100 & -- & -- & 100 & -- & -- & 100\\
cy (N) & -- & 40 & 45 & -- & 40 & 45 & -- & 40 & 45\\
sy (N) & -- & 160 & 180 & -- & 160 & 180 & -- & 160 & 180\\
\midrule
SE & HAC & HAC & HAC &HAC &HAC &HAC &HAC &HAC &HAC \\
Observations & 800 & 800 & 900 & 798 & 798 & 898 & 798 & 798 & 898 \\ 
R$^{2}$ & 0.392 & 0.588 & 0.729 & 0.763 & 0.924 & 0.899 & 0.858 & 0.951 & 0.924 \\ 
Adjusted R$^{2}$ & 0.387 & 0.584 & 0.727 & 0.762 & 0.923 & 0.898 & 0.857 & 0.950 & 0.924 \\ 
Residual SE & 0.068 & 0.026 & 0.033 & 0.055 & 0.019 & 0.026 & 0.040 & 0.016 & 0.023 \\ 
IRLS Coverged & Yes & Yes & Yes & Yes & Yes & Yes & Yes & Yes & Yes \\ \bottomrule \\[-1em]
\multicolumn{7}{l}{\small \textit{Note:} The dependent variable is the natural log of VA, which is regressed on the  export shares}   & \multicolumn{3}{r}{$^{*}$p$<$0.1; $^{**}$p$<$0.05; $^{***}$p$<$0.01} \\ [-0.2em]
\multicolumn{10}{l}{\small \quad \quad \quad (1)-(3), natural log of the shares (4)-(6), and natural log of values (7)-(9) of I2E and E2R.} \\
\end{tabular} 
}
\end{table} 
\FloatBarrier

In comparison to the least squares estimates in Table \ref{tab:VAGRREG}, the coefficients in Table \ref{tab:VAGRREG_R} spread the effect more evenly across the lags in the FD specification, and also let the FD-FE and FE specifications move closer to the plain FD specification. Notably, the coefficient in L1.I2E in the log-level specification is now positive and significant. The results for the FD S and FD ES specifications imply that a 0.01 / 1\% increase in I2E results in a 0.8\% / 0.27\% increase in VA after two years. The coefficients on the FD E specification are slightly smaller than in Table  \ref{tab:VAGRREG}, with a 1\% increase in the value of I2E resulting in an 0.11\% increase in VA within two years. Similar distributional effects hold for E2R, where additionally the combined semi-elasticity from FD S is around 2 and thus a bit larger than the 1.86 in Table \ref{tab:VAGRREG}, while the elasticities are a bit smaller with 0.21 / 0.082 from FD ES / FD E compared to 0.22 / 0.1 in Table \ref{tab:VAGRREG}. In terms of robustness, it should also be noted here that multicollinearity between the various lagged values is low, at a maximum VIF of 1.3 in all models. 

\subsubsection{Manufacturing Estimates}

Tables \ref{tab:VAGRREG_MAN} and \ref{tab:VAGRREG_MAN_R} reports equivalent regressions run for the manufacturing sub-sample of sectors. Compared to Tables \ref{tab:VAGRREG} and \ref{tab:VAGRREG_R}, the results appear to be broadly similar, and the coefficients are slightly larger. 

\begin{table}[h!]
\centering
\caption{\label{tab:VAGRREG_MAN} \textsc{Value Added Regressions: Manufacturing}}
\resizebox{\textwidth}{!}{
\begin{tabular}[t]{lccccccccc} \toprule
% Dependent Variable:&\multicolumn{6}{c}{log\_VA}\\
 \textit{Model:}    & FD S & FD-FE S & FE S & FD ES & FD-FE ES & FE ES & FD E & FD-FE E & FE E\\
                             &(1) & (2) & (3) & (4) & (5) & (6) & (7) & (8) & (9)\\ \midrule 
                 &&&&&&&&&  \\
(Intercept)&0.0678$^{***}$ &    &    & 0.0752$^{***}$ &    &    & 0.0419$^{***}$ &    &   \\
  &(0.0088) &    &    & (0.0038) &    &    & (0.0041) &    &   \\\\
I2E&-1.488$^{**}$ & -5.526$^{***}$ & -5.952$^{**}$ & -0.1546$^{*}$ & -0.8395$^{***}$ & -0.6364$^{**}$ & -0.5477$^{***}$ & -0.4025$^{***}$ & -0.3972$^{***}$\\
  &(0.5558) & (1.892) & (2.449) & (0.0901) & (0.2898) & (0.2812) & (0.0491) & (0.0599) & (0.0954)\\\\
L1.I2E&-0.9133 & -3.216 & -5.472 & 0.1360$^{***}$ & -0.3627$^{**}$ & -0.6820$^{***}$ & 0.0942$^{*}$ & -0.0548 & -0.0986$^{*}$\\
  &(0.5808) & (3.368) & (5.684) & (0.0337) & (0.1647) & (0.1900) & (0.0466) & (0.0388) & (0.0567)\\\\
L2.I2E&0.5380$^{**}$ & 2.480 & 6.195 & 0.1782$^{***}$ & 0.3872 & 0.4091$^{*}$ & 0.0590$^{**}$ & 0.0175 & 0.1184$^{**}$\\
  &(0.2403) & (1.625) & (4.257) & (0.0506) & (0.2602) & (0.2211) & (0.0252) & (0.0249) & (0.0568)\\\\
E2R&3.990$^{**}$ & 0.4527 & 1.434 & 0.9845$^{***}$ & 0.8140$^{***}$ & 0.8362$^{***}$ & 0.8882$^{***}$ & 0.9013$^{***}$ & 0.8926$^{***}$\\
  &(1.479) & (0.4625) & (1.001) & (0.0505) & (0.0867) & (0.0643) & (0.0469) & (0.0456) & (0.0478)\\\\
L1.E2R&2.446$^{***}$ & -0.2512 & -0.8829 & 0.2056$^{**}$ & 0.0725 & 0.1013 & 0.1228$^{**}$ & 0.0184 & 0.0883\\
  &(0.4453) & (0.4383) & (0.9402) & (0.0771) & (0.0519) & (0.0837) & (0.0455) & (0.0230) & (0.0838)\\\\
L2.E2R&0.2386 & -0.3984 & -0.8486 & 0.0417 & -0.0192 & -0.1399 & -0.0097 & 0.0252 & -0.1072\\
  &(0.2851) & (0.3696) & (0.5726) & (0.0337) & (0.0230) & (0.0908) & (0.0197) & (0.0281) & (0.0894)\\\\
\midrule \emph{Fixed-Effects:} &   &   &   &   &   &  \\
cs (N) & -- & -- & 40 & -- & -- & 40 & -- & -- & 40\\
cy (N) & -- & 40 & 45 & -- & 40 & 45 & -- & 40 & 45\\
sy (N) & -- & 64 & 72 & -- & 64 & 72 & -- & 64 & 72\\
\midrule
% \emph{Fit statistics}&  & & & & & \\
Cluster SE & cs & cs cy sy & cs cy sy & cs & cs cy sy & cs cy sy & cs & cs cy sy & cs cy sy\\
Observations & 320&320&360&320&320&360&320&320&360\\
R$^2$ & 0.309&0.837&0.995&0.813&0.979&0.999&0.849&0.987&1.00\\
Within R$^2$ & &0.173&0.178&&0.892&0.925&&0.936&0.940\\ \bottomrule \\[-1em]
\multicolumn{7}{l}{\small \textit{Note:} The dependent variable is the natural log of VA, which is regressed on the  export shares}   & \multicolumn{3}{r}{$^{*}$p$<$0.1; $^{**}$p$<$0.05; $^{***}$p$<$0.01} \\ [-0.2em]
\multicolumn{10}{l}{\small \quad \quad \quad (1)-(3), natural log of the shares (4)-(6), and natural log of values (7)-(9) of I2E and E2R.} \\
%\multicolumn{7}{l}{\textsuperscript{} * p < 0.1, ** p < 0.05, *** p < 0.01}\\
%\multicolumn{7}{l}{\emph{Signif. Codes: ***: 0.01, **: 0.05, *: 0.1}}\\
\end{tabular}
}
\end{table}
\FloatBarrier



Again especially the log-level specification is affected by outliers, thus under general considerations the robust estimates in Table \ref{tab:VAGRREG_MAN_R} are preferable, and also distribute the effect more evenly across lagged coefficients. In both Tables \ref{tab:VAGRREG_MAN} and \ref{tab:VAGRREG_MAN_R}, the lagged coefficients on the FD-FE and FE specifications appear to be attenuated and are largely insignificant. Gauging thus from the simple first-difference lagged coefficients in  Table \ref{tab:VAGRREG_MAN_R}, the semi-elasticity of manufacturing VA to a 0.01 unit increase in I2E / E2R is a 0.58\% / 2.47\% increase, the elasticity w.r.t. a 1\% increase in I2E / E2R is 0.28\% / 0.31\%, and the elasticity w.r.t. a 1\% increase in the values of I2E / E2R is 0.15\% / 0.07\%. Again these are lower bound effect sizes, \textit{ceteris paribus} any contemporaneous effects. \newline



\subsubsection{Comparison of Estimates}
In summary, Tables \ref{tab:VAGRREG_MAN} and \ref{tab:VAGRREG_MAN_R} present similar results to Tables \ref{tab:VAGRREG} and \ref{tab:VAGRREG_R}. Using only the robust estimates from Tables \ref{tab:VAGRREG_R} and \ref{tab:VAGRREG_MAN_R} of the preferred FD specification implies that:
\begin{itemize}
\item A 0.01 unit increase in I2E / E2R yields a 0.81\% / 1.97\% increase in overall VA and a 0.58\% / 2.47\% increase in manufacturing VA after 2 years.
\item A 1\% increase in I2E / E2R yields a 0.27\% / 0.21\% increase in overall VA and a 0.28\% / 0.31\% increase in manufacturing VA after 2 years.
\item A 1\% increase in the values of I2E / E2R yields a 0.11\% / 0.082\% increase in overall VA and a 0.15\% / 0.07\% increase in manufacturing VA after 2 years.
\end{itemize}

Given that, as is evident from Figures \ref{fig:GROWTH_REG_TS} and \ref{fig:GROWTH_REG_Hists}, manufacturing sectors have a higher I2E and lower E2R ratio than other sectors, and, as shown in Table \ref{tab:SUMM_GROWTH}, overall I2E in the sample with an average of 0.2 is higher than E2R with an average of 0.15, it appears natural that the semi-elasticity of VA w.r.t. E2R is higher than w.r.t. I2E, and that manufacturing sectors on average gain more from improvements in E2R. 

% Table created by stargazer v.5.2.2 by Marek Hlavac, Harvard University. E-mail: hlavac at fas.harvard.edu
% Date and time: Sat, May 08, 2021 - 12:40:45 AM
\begin{table}[!htbp] \centering 
\caption{\label{tab:VAGRREG_MAN_R} \textsc{Value Added Regressions: Manufacturing: Robust MM Estimates}}
\resizebox{\textwidth}{!}{
\begin{tabular}[t]{lccccccccc} \toprule
% Dependent Variable:&\multicolumn{6}{c}{log\_VA}\\
 \textit{Model:}    & FD S & FD-FE S & FE S & FD ES & FD-FE ES & FE ES & FD E & FD-FE E & FE E\\
                             &(1) & (2) & (3) & (4) & (5) & (6) & (7) & (8) & (9)\\ \midrule 
                 &&&&&&&&&  \\
 (Intercept) & 0.0718$^{***}$ & -0.0038$^{**}$ & -0.0040 & 0.0728$^{***}$ & 0.0006 & -0.0002 & 0.0310$^{***}$ & 0.0007 & -0.0004 \\ 
  & (0.0045) & (0.0015) & (0.0024) & (0.0038) & (0.0010) & (0.0013) & (0.0034) & (0.0008) & (0.0012) \\ 
  & & & & & & & & & \\ 
 I2E & -0.5597 & -2.4579$^{***}$ & -5.9545$^{***}$ & -0.0545 & -0.8506$^{***}$ & -0.7222$^{***}$ & -0.5498$^{***}$ & -0.4258$^{***}$ & -0.4078$^{***}$ \\ 
  & (0.3580) & (0.5795) & (1.1946) & (0.0678) & (0.1373) & (0.1886) & (0.0430) & (0.0408) & (0.0832) \\ 
  & & & & & & & & & \\ 
 L1.I2E & 0.1038 & 0.3573 & 0.8177 & 0.1185$^{***}$ & -0.2473$^{**}$ & -0.4732$^{*}$ & 0.0835$^{**}$ & -0.0626$^{*}$ & -0.0972 \\ 
  & (0.2566) & (0.6005) & (1.1756) & (0.0455) & (0.1252) & (0.2637) & (0.0377) & (0.0362) & (0.0654) \\ 
  & & & & & & & & & \\ 
 L2.I2E & 0.4707$^{**}$ & 0.8345$^{**}$ & 1.4843$^{**}$ & 0.1592$^{***}$ & 0.2528$^{*}$ & 0.1105 & 0.0639$^{***}$ & -0.0153 & 0.0581$^{*}$ \\ 
  & (0.2376) & (0.4161) & (0.7338) & (0.0458) & (0.1407) & (0.1830) & (0.0195) & (0.0166) & (0.0320) \\ 
  & & & & & & & & & \\ 
 E2R & 6.2357$^{***}$ & 0.1768 & 0.8435$^{**}$ & 0.9901$^{***}$ & 0.8083$^{***}$ & 0.8488$^{***}$ & 0.9322$^{***}$ & 0.8903$^{***}$ & 0.9058$^{***}$ \\ 
  & (0.5858) & (0.1489) & (0.3522) & (0.0465) & (0.0460) & (0.0543) & (0.0371) & (0.0235) & (0.0331) \\ 
  & & & & & & & & & \\ 
 L1.E2R & 1.6517$^{***}$ & -0.0723 & -0.2776 & 0.2336$^{***}$ & 0.0246 & 0.0897 & 0.0690$^{*}$ & 0.0109 & 0.0729$^{**}$ \\ 
  & (0.3458) & (0.1590) & (0.4348) & (0.0441) & (0.0215) & (0.0712) & (0.0409) & (0.0136) & (0.0336) \\ 
  & & & & & & & & & \\ 
 L2.E2R & 0.8143$^{***}$ & -0.1095 & -0.5303$^{*}$ & 0.0731$^{***}$ & -0.0168 & -0.0802 & -0.0052 & 0.0166 & -0.0416 \\ 
  & (0.2503) & (0.1171) & (0.3107) & (0.0210) & (0.0350) & (0.0919) & (0.0199) & (0.0121) & (0.0259) \\ 
  & & & & & & & & & \\ 
\midrule \emph{Fixed-Effects:} &   &   &   &   &   &  \\
cs (N) & -- & -- & 40 & -- & -- & 40 & -- & -- & 40\\
cy (N) & -- & 40 & 45 & -- & 40 & 45 & -- & 40 & 45\\
sy (N) & -- & 64 & 72 & -- & 64 & 72 & -- & 64 & 72\\
\midrule
SE & HAC & HAC & HAC &HAC &HAC &HAC &HAC &HAC &HAC \\
Observations & 320 & 320 & 360 & 320 & 320 & 360 & 320 & 320 & 360 \\ 
R$^{2}$ & 0.539 & 0.136 & 0.299 & 0.871 & 0.907 & 0.956 & 0.936 & 0.944 & 0.961 \\ 
Adjusted R$^{2}$ & 0.530 & 0.119 & 0.287 & 0.869 & 0.905 & 0.955 & 0.934 & 0.942 & 0.960 \\ 
Residual SE & 0.066 & 0.024 & 0.039 & 0.049 & 0.016 & 0.021 & 0.038 & 0.013 & 0.020 \\ 
IRLS Coverged & Yes & Yes & Yes & Yes & Yes & Yes & Yes & Yes & Yes \\ \bottomrule \\[-1em]
\multicolumn{7}{l}{\small \textit{Note:} The dependent variable is the natural log of VA, which is regressed on the  export shares}   & \multicolumn{3}{r}{$^{*}$p$<$0.1; $^{**}$p$<$0.05; $^{***}$p$<$0.01} \\ [-0.2em]
\multicolumn{10}{l}{\small \quad \quad \quad (1)-(3), natural log of the shares (4)-(6), and natural log of values (7)-(9) of I2E and E2R.} \\
\end{tabular} 
}
\end{table} 
\FloatBarrier

Larger productivity gains from forward itegration, proxied by E2R, is also a salient feature in the literature. \citet{Kummritz20161}, using a sample of mostly manufacturing industries, finds robust benefits of GVC backward and forward integration on VA in both developing and developed / middle income countries, with a larger benefit of forward integration (E2R) at elasticities as high as 0.58 for developing / middle income countries and 0.68 for developed countries, and I2E elasticities smaller around 0.2-0.3. He also estimates labour productivity elasticities to E2R of 0.29 for developing / middle-income countries and 0.49 for developed countries. \newline 

In a similar exercise \citet{kummritz2015global} finds that high-income countries benefit relatively more from forward linkages (E2R) whereas middle-income countries also benefit from backward linkages (I2E). The results presented here suggest that EAC countries could benefit almost equally from increases in backward and forward GVC integration, with manufacturing sectors drawing slightly greater benefits from forward integration. 
%The elasticity  of VA to I2E within two years time is around $0.3$ for both the manufacturing sectors and all other sectors taken together, suggesting a gain from foreign technology in production. When looking at forward GVC integration (E2R), all sectors together have an cumulative growth elasticity of around $0.11-0.15$, but the manufacturing sectors have an elasticity of $0.25-0.3$, which is more than twice as large. Thus manufacturing sectors in the EAC benefit equally from backward and forward GVC integration, whereas other sectors benefit mostly from backward GVC integration. 
This is a sensible result, as for example increased forward integration in primary products like agriculture or mining is not necessarily associated with domestic productivity gains, but if manufactured exports are re-exported as part of a GVC, they likely have to be of sufficient quality.  \newline 




\section{Summary}

This study set out to undertake a broad and comprehensive analysis of the patterns of global and regional value chain integation in the EAC, using global MRIO tables provided by EORA - the only ICIO database covering the EAC at the time of writing. The decade 2005-2015 was chosen as analysis period, mandated by the full EAC membership of Rwanda and Burundi in 2007, and the availability of data only up to 2015. \newline

The paper commenced with a broad exposition of gross input-output (IO), exports and value added (VA) flows within the EAC, and between the EAC and different world regions. This exposition was followed by an econometric analysis seeking to quantify the contribution of GVC integation to member countries GDP growth and industrialization. The results and discussion are summarized below. \newline

\subsection{Gross Flows}

The general structure of production in most EAC countries is that domestic VA (GDP) constitutes about 50-60\% of gross output. The remainder of gross output are intermediate inputs, of which in most EAC countries 5-10\% are imported. In all EAC countries apart from Kenya, 5\% or less of  gross output are exported by 2015, in Kenya about 8\%. In Uganda, about 25\% of imported inputs are from the EAC, followed by Rwanda at 14\%, and the other countries below 10\% (Kenya around 2\%). In 2015, about 30\% of Ugandan and Kenyan exports are to other EAC members, whereas the EAC export shares of all other members are below 7\%. \newline

The analysis of gross IO flows highlighted the supplier role of Kenya within the EAC, followed, with some distance, by Uganda and Tanzania, and a negligible role of Rwanda and Burundi. The largest external supplier of inputs to all EAC countries is the European Union, followed, again with some distance, by South Asia (SAS), the Middle East \& North Africa (MEA), the rest of Sub-Saharan Africa (SSA) and China. \newline

The EAC trade balance in gross IO flows is negative, and has deteriorated over time: In 2005 intermediate imports were 2x larger than intermediate exports, by 2015 they were 3x larger. \newline

At the sector-level, the largest outgoing flows are agricultural inputs, expecially from Kenya, but also from Tanzania and Uganda, into EU food processing and beverage industries, and agricultural re-exports by the EU. Among the incoming flows, the EU, and also MEA, SAS and SSA, supply inputs for EAC transport, construction, petro-chemical and electrical machinery sectors. Inside the EAC, Kenya supplies mining and manufcaturing inputs to Ugandan and Tanzanian manufacturing (esp. petro-chemicals and electrical machinery), transport and construction sectors, and Uganda supplies agricultural inputs to the Kenyan food processing industry.  At a significantly lower scale, Uganda also supplies manufacturing inputs for Rwandan manufacturing (esp. petro-chemicals, metal products and electrical machinery). Tanzania has much less of a supplier role, the only flow close to 1 million dollars in 2015 is Tanzanian mining inputs for the Ugandan petro-chemical industry. Rwanda and Burundi appear to be negligible as suppliers of inputs in EAC production, and South Sudan appears not to be economically integrated with the EAC on the production side at all. \newline

Over the 2005-2015 period, ROW inputs into EAC production are quite steadily 12 times greater than EAC imputs into EAC production, whereas intermediate outflows to the ROW are only 4.4 times greater  than inner-EAC flows in 2015, down from 5.9 times greater in 2005. This suggests an asymmetric development where the EAC is integrating but only in terms of intermediate outflows. \newline

In terms of exports, in all major EAC countries more than 50\% of exports is made up from agriculture, fishing, mining, and food and beverage industries. Uganda has the largest share of raw agricultural exports at 38\% in 2015, Kenya and Tanzani have lower shares at 29\% and 24\%, respectively. In Kenya the food and beverages industry is quite important, comprising 14.3\% of exports in 2015. Rwanda has a large share of 24\% in mining exports. Apart from primary products, Kenya also exports petro-chemicals (10.5\%) and electrical machinery (6\%). Tanzania also exports textiles and other manufactures (both at shares of 9\%). For all EAC countries exported transport services also have a share between 5 and 10\%. \newline

Of the EAC members, only Uganda and Kenya export around 30\% of their exports to the EAC. Ugandan exports to the EAC are comprised to more than 60\% of primary agriculture. More than 50\% of Kenyan exports to the EAC are manufactured products such as petro-chemical, metal products and electrical machinery. Tanzania exports 7\% to the EAC, more than 20\% of which is foods and beverages. Rwanda, Burundi and South Sudan export less than 2\% to the EAC.  \newline 

\subsection{Value Added Flows}

In VA terms, deriving the value-added-multiplier matrix \textbf{VB} at both aggregate and sectoral levels yielded that, in the aggregate, 85-90\% of the value of aggregate output is sourced within the country for most EAC countries. Notable exceptions were Tanzania at 68\% of domestic VA and South Sudan at 98\% domestic VA (in 2015). \newline

As largest foreign supplier of VA in 2015, the EU adds 2.38\% to Uganda's, 7.49\% to Tanzania's, 4.58\% to Kenya's 3.77\% to Rwanda's, 2.77\% to Burundi's and 0.11\% to South Sudan's production. Other significant suppliers of VA in EAC production are South Asia, Sub-Saharan Africa, the Middle East and China. Among the EAC countries, Kenya takes on a significant role, supplying 2.24\% in Uganda's, 1.7\% in Tanzania's and 0.83\% in Rwanda's production. Uganda also adds 0.38\% in Rwanda's production. \newline

The foreign VA share in a countries final goods production (and exports) was termed vertical specialization (VS) by \citet{hummels2001nature}, and I2E by \citet{baldwin2015supply}, and is a widely used measure of backward integration into GVCs. Computing this measure for the years 2005-2015 yielded little evidence that EAC countries have increased their participation in GVCs. In Uganda, VS has been fluctuating between 10\% and 12\% without a clear trend. In Tanzania, VS has increased remarkably from 13\% in 2005 to 29\% in 2015, but this result should be treated with extreme caution as the macro-data for Tanzania is highly distorted\footnote{Tanzanian GDP for example is decreasing as shown in Figure \ref{fig:VAexp}, which calls into question the reliability of IO relationships and shares computed for Tanzania.}. I Kenya, VS has been oscillating around 15\%, in Rwanda around 12\%, in Burundi around 10\%. A notable feature of the data is that VS appears to have increased in Uganda, Kenya, Rwanda and Burundi in the years 2005-2011, and decreased again from 2012-2015. \newline

In most EAC countries, EAC VA shares (in particular VA by Kenya and Uganda) have increased by small amounts of 0-0.5 percentage points over the analyzed period, indicating that the relative importance of EAC neighbours as suppliers of inputs has increased. \newline

 At the sector level, VS is highest in manufacturing sectors in all EAC countries, coming as high as 60\% in some Tanzanian sectors (which should be regarded with caution). In Uganda, Kenya, Rwanda and Burundi, between 20\% and 40\% of VA in core manufacturing industries such as petro-chemicals, electrical machinery, metal products and transport equipment is foreign sourced. Recycling and re-export industries in a number of EAC countries also have high VS of 70\% and above. Again most of the foreign VA in EAC production comes from the EU, which also quite homogeneously applies at the sector level. \newline
 
The \textbf{VB} matrix was also used to decompose IO flows into VA terms, yielding a more favorable but still very negative EAC trade balance in intermediates. This is a result of IO flows from ROW to the EAC manufacturing industries having less domestic VA, than EAC intermediate exports to ROW comprising mostly of primary inputs. Also in VA terms intermediate flows inside the EAC are around 14 times smaller than intermediate inflows from ROW to EAC. Progress towards greater regional integration in intermediates has only been made on the outflow side, with the relative
importance of ROW declining from 7.4 times greater in 2005 to 5.5 times greater in 2015. \newline 

VA exports (comprising of both intermediate ouflows and final goods) increased in all EAC countries between 2005 and 2015, with Ugandan VA exports estimated at 1 billion USD at basic prices in 2015, and Kenyan VA exports around 6 billion. \newline 

A popular measure of forward GVC integration is the VA share of gross exports that enters foreign countries exports, termed VS1 by \citet{hummels2001nature} and E2R by \citet{baldwin2015supply}. In most EAC countries VS1 has been quite stable over time, in Uganda is has dropped from 14\% in 2005 to 12\% in 2015, in Kenya it has been stable at 12.5\%, in Rwanda it dropped from 23\% to 16\%, in Burundi it is stable around 17\%. Thus overall the EAC GVC situation appears quite stable, no country has progressed to supply significant value for foreign export production. \newline

At the sector level, mostly primary inputs and wholesale traded goods are re-exported by receiving countries. In Uganda, 48\% of mining exports are re-exported, and 30\% of wholesale goods exports. In Kenya, Rwanda, and Burundi, apart from wholesale trade, 20\% and more of agricultural exports are being re-exported. At the sector level, some development is visible in the 2005-2015 period: most manufacturing sectors increased their imported content (VS) and decreased their re-exported content (VS1), whereas in most primary sectors (mainly agriculture), the opposite is the case. Overall this suggests that EAC countries have failed to move upsteam into manufacturing GVCs, and are increasingly concentrating on agriculture and the production of simple manufactures for domestic consumption and final export. \newline

On the regional frontier, computing the EAC partner contribution to each members VS, VS1, value-added imports (VAI) and imports of final goods (VAFI) yielded moderate declines in most cases. The only marked increase was the EAC share in Kenya's VS1, which increased from 5.7\% in 2005 to 7.4\% in 2015, highlighting Kenya's increasing supplier role for EAC export production. It should be noted that these shares are at very different levels for different EAC countries. In Uganda, the EAC accounts for 21\% of VS, 6\% of VS1 and 17\% of VAI. Thus Uganda primarily relies on the EAC for inputs, whereas Kenya, with an EAC share in VS of only 0.9\%, primarily exports inputs to EAC countries. \newline

Thus in overall terms, with the exception on Tanzania and Rwanda, which seem to have become globally more integrated as users of inputs for export production (VS), and Kenya which has increased it's role as an EAC supplier of inputs (EAC share in VS1), it appears that the decade under analysis has seen little progress towards integration into global or regional value chains.  \newline

Advanced analysis of GVC integration using the \citet{koopman2014tracing} (short KWW) decomposition of gross exports yielded that there is hardly any double counted domestic VA in gross exports, which arises from two-way trade in intermediate goods and consititutes a meaningful fraction of exports in advanced economies. This implies that EAC countries only engage in uni-directional (and mostly shorter) GVCs. Nevertheless, up to 10\% of VA exported is foreign double counted, implying the re-importing and exporting of foreign VA is prevalent. There is also practically 0 domestic VA that returns via final or intermediate imports, implying again shallow GVC integration and relatively uni-directional trading relationships. \newline
 
 Indices of upstreamness as the share of domestic VA in intermediate exports over total DVA in exports, and downstreamness as foreign VA in final exports over total FVA in exports, showed that, apart from Tanzania where the data suggest a slight upstream movement, all EAC countries have moved downstream in GVCs, with both less domestic content going into intermediate exports and more FVA going to final goods exports. \newline
 
 Using KWW to decompose EAC exports to the EAC, yields more domestic and foreign VA in final goods compared to EAC exports to ROW. Also double counted and re-exported components are lower, confirming that EAC countries engage more in GVCs with ROW than with their EAC neighbours. \newline
 
 Using the composition developed by \citet{wang2013quantifying}, exports were also decomposed at the sector level. At the sector level it is again evident that manufacturing sectors have higher VS shares than other sectors, and also higher amounts of double counting, although mostly of foreign VA. Calculations of upstreamness and downstreamness ratios at the sector level suggest that nearly all setors in all EAC countries have moved downstream between 2005 and 2015. This is surprising, indicating that in terms of GVC integration there are little exceptions or high-performing sectors, and the downstream trends observed in the aggregate dissipate quite uniformly to the sector level. \newline 

A final exercise in the exploration of VA flows was to compute the New Revealed Comparative Advantage (NRCA) index,  using GDP in exports as proposed by \citet{koopman2014tracing}. This revealed a comparative advantage in agriculture and fishing, and a comparative disadvantage in manufacturing for all EAC members. On the services side, in all members apart from Kenya, constuction, maintenance and repair activities, accommodation and food services (including Kenya) and activities of private housholds (self-employment) also have quite a strong NRCA. Kenya has a notable comparative advantage in food and beverages. \newline

In the 2005-2015 period, NRCA in agriculture has remained stable, whereas all members lost NRCA in mining. All members also have gained strongly in re-exporting goods.  Other country and sector developments are more heterogenous. Uganda and Tanzania have gained NRCA in fishing, Kenya notably has gained NRCA in all manufacturing sectors. \newline

To better expose potential issues and challenges of regional integration, NRCA has also been computed for EAC members relative to the combined EAC export mix. The core result of this was that relative to other EAC members, Uganda has a NRCA in agriculture, Tanzania in fishing, Rwanda in mining, and Kenya in core manufacturing sectors such as wood and paper, petro-chemicals, metal products and electrical machinery as well as food and beverages. Furthermore, Rwanda and Burundi, and to a weaker extent Uganda, have a NRCA in construction, maintenance and repairs, wholesale and retail trade, whereas Tanzania appears to have a NRCA in other manufacturing, recycling, and financial and business services.

\subsection{GVCs and Industrial Development} 

The exposition of gross, and particularly of VA flows, suggests that GVC integration has been sluggish, and even declined in most EAC countries. The nature of integration has also shifted towards more downstream production resulting in the export of final goods using domestic and foreign VA. Since previous research has found GVCs to be beneficial for growth, particularly in advanced and middle income countries, and since most transfer of technology is regarded as taking place in the upstream parts of the value chains (rather than downstream assembly tasks or supplying primary inputs), it remains to investigate whether increasing GVC integration could benefit the EAC. \newline

Towards this end an econometric analysis was coducted regressing VA (GDP) on I2E (VS) and E2R (VS1) at the sector level. This analysis rested on the assumptions that sector-level heterogeneity can provide some information about the likely benefit of GVCs on aggregate economic activity, and that changes in I2E and E2R benefit economic activity with a lag of up to two years (which is the main identification assumptions apart from the use of first differences and fixed effects to eliminate cross-sectional heterogeneity). Robust statistical methods were used to minimize the effects of outliers in individual sectors on the result. Various effect sizes were obtained by regressing the log of VA on the export shares, the log of the shares, and the log of the values of I2E and E2R. Furthermore, additional regressions where run only for a subsample of core manufacturing sectors to obtain an estimate of the productivity impact of GVC integration on manufacturing. \newline

The coefficients from all specifications show a strong negative contemporaneous relationship between VA and I2E and a strong positive relationship between VA and E2R. These contemporaneous relationships were however regarded as mechanical and affected by reverse-causalities between GDP and the GVC indicators, and hence disregarded in favor of the coefficients on the lags of I2E and E2R, which were carefully interpreted as causal. \newline

The lagged relationships, if causal, give a lower bound effect size estimate conditional on the contemporaneous relationship. Results from robust estimation of the preferred first-difference specification\footnote{First-differences was preferred over fixed effects due to significant serial correlation in the error term.} show a moderate positive effect of both I2E and E2R on VA. About 70\% of this effect materializes after one year, while the remaining approx. 30\% materializes in the second year following the increase. Estimation of the elasticity of VA w.r.t. the I2E and E2R shares yielded that a 1\% increase in I2E produces a 0.27\% increase in VA after two years and a 1\% increase in E2R yields a 0.21\% increase in VA after two years. For manufacturing sectors the elasticity to I2E is 0.28\% and to E2R is 0.31\%. The overall larger impact on manufacturing, and particularly the importance of forward GVC integration as measured by E2R, emphasizes the potential of GVCs to boost productivity in EAC manufacturing sectors. \newline

Larger gains from forward integration were also estimated in the literature such as \citet{Kummritz20161} with manufacturing-heavy samples of sectors in OECD ICIO tables. The elasticities on forward GVC integration in these high-income countries is higher at around 0.6, whereas backup GVC integration (I2E) has a significantly lower elasticity (0.1-0.3) in high-income countries. Another salient finding by \citet{kummritz2015global} is that low- and middle-income countries generally benefit less from GVC integration, but benefit relatively more from backward linkages (I2E) compared to high-income countries. These findings appear to be broadly confirmed by the empirical results of this paper. \newpage

\section{Conclusion} 

Inspite of severe data limitations for EAC countries, in particular for Tanzania, by virtue of using a global MRIO model where data inconsistencies in small countries data are scaled away and time series interpolation is used to fill gaps in the data, this analysis has produced a few viable and important findings that are broadly consumerable with the observable EAC production and trading patterns, and  with GVC analysis conducted for other low-and middle-income countries covered by higher quality ICIO tables such as OECD-TiVA and WIOD. \newline

The most important of these findings is that in the years 2005-2015, EAC members do not seem to have integrated much further into GVCs, both in overall terms and in terms of production sharing within the EAC. Moderate overall developments are visible in Rwanda and Tanzania which have gradually increased the foreign content in their production (VS), and on the regional level Kenya has become an important supplier of inputs to its EAC neighbours, which also feed into export production of most EAC members (E2R). Yet inputs to EAC production from the rest of the world remain consistency 12-14 times greater in VA terms than inputs from EAC neighbours. \newline

Furthermore, there appears to be a downstream shift in existing GVC relationships, with more domestic and foreign VA going into final goods production for domestic use and final export, while maintaining high levels of primary input exports such as agriculture. This movement comes at the cost of producing high-quality intermediate inputs which would allow EAC countries to integrate in upstream parts of GVCs where most efficiency gains and use of more complex technology occur. This downstream shift appears to be prevalent across EAC sectors, including manufacturing. \newline

Econometric analysis provided evidence that increased GVC integation can benefit growth in the EAC. The elasticities of domestic VA to the two main GVC ratios  (I2E and E2R) are around 0.25, which is a lower-bound estimate derived from the 1- and 2-year lagged coefficients in reduced from panel data models.  This impact estimate is slightly higher for manufacturing, with elasticity estimates around 0.3. Manufacturing sectors would also benefit slightly more from forward integration (E2R), in-line with the global evidence (e.g.  \citet{Kummritz20161}, \citet{kummritz2015global}). \newline

These findings suggest that policies that would help EAC industries to integrate more into regional and global value chains, through the production of high-quality intermediate inputs, would benefit EAC growth and productivity in the medium to long term. \newline 

It remains at this point to conduct further research to confirm and investigate the nature and causes of the observed downstream shift towards final goods in manufacturing observed by this study. It also remains to research further the more general limitations inhibiting EAC industries to engage in GVCs and RVCs, relating e.g. to structural factors, the business environment and trade policies. \newline

Finally, a lot more research can be done into the effects of GVCs on productive restructuring in the EAC and Sub-Saharan African economies more generally. This includes effects on product and export diversification and competitiveness, labour productivity and employment, and the specific details of where and how much value is added in different sectors as they become permeated by GVCs. Filling in these knowledge gaps through focused research will build a foundation for informed and effective GVC policies in the EAC. 



\newpage
\bibliographystyle{apacite}
\bibliography{GVC}


\section*{Appendix}

\subsection*{Countries and Regions}

\begin{table}[h!]
\centering
\caption{\textsc{Countries and Regions}}

\label{tab:ctrydet}
\vspace{2mm}
\resizebox{0.85\textwidth}{!}{
\begin{tabular}{llp{6cm}} \toprule
\textit{Region} & \textit{Description} & \textit{Countries} \\ \midrule
EAC & East African Community & UGA, TZA, KEN, RWA, BDI, SSD \\ \\
SSA & Sub-Saharan Africa (Excluding EAC) & AGO, BEN, BFA, BWA, CAF, CIV, CMR, COD, COG, COM, CPV, ERI, ETH, GAB, GHA, GIN, GMB, GNB, GNQ, LBR, LSO, MDG, MLI, MOZ, MRT, MUS, MWI, NAM, NER, NGA, SDN, SEN, SLE, SOM, STP, SWZ, SYC, TCD, TGO, ZAF, ZMB, ZWE \\ \\
EUU & European Union + GBR & AUT, BEL, BGR, CYP, CZE, DEU, DNK, ESP, EST, FIN, FRA, GBR, GRC, HRV, HUN, IRL, ITA, LTU, LUX, LVA, NLD, POL, PRT, ROU, SVK, SVN, SWE, MLT \\ \\
ECA & Europe and Central Asia (Non-EU) & ALB, AND, ARM, AZE, BIH, BLR, CHE, CHI, FRO, GEO, GIB, GRL, IMN, ISL, KAZ, KGZ, LIE, MCO, MDA, MKD, MNE, NOR, RUS, SMR, SRB, TJK, TKM, TUR, UKR, UZB, XKX \\ \\
MEA & Middle East and North Africa & ARE, BHR, DJI, DZA, EGY, IRN, IRQ, ISR, JOR, KWT, LBN, LBY, MAR, OMN, PSE, QAT, SAU, SYR, TUN, YEM \\ \\
NAC & North America and Canada & BMU, CAN, USA \\ \\
LAC & Latin America and Carribean & ABW, ARG, ATG, BHS, BLZ, BOL, BRA, BRB, CHL, COL, CRI, CUB, CUW, CYM, DMA, DOM, ECU, GRD, GTM, GUY, HND, HTI, JAM, KNA, LCA, MAF, MEX, NIC, PAN, PER, PRI, PRY, SLV, SUR, SXM, TCA, TTO, URY, VCT, VEN, VGB, VIR \\ \\
ASE & ASEAN & BRN, IDN, KHM, LAO, MMR, MYS, PHL, SGP, THA, VNM \\ \\
SAS & South Asia & AFG, BGD, BTN, IND, LKA, MDV, NPL, PAK \\ \\
CHN & China & CHN, HKG, TWN \\ \\
ROA & Rest of Asia & ASM, GUM, JPN, KOR, MAC, MNG, MNP, NCL, PRK, PYF, TLS \\ \\
OCE & Oceania & AUS, FJI, FSM, KIR, MHL, NRU, NZL, PLW, PNG, SLB, TON, TUV, VUT, WSM
 \\ \bottomrule
\end{tabular}
}
\end{table}
\FloatBarrier

\subsection*{Examination of the Data}
To examine the data, I compute global GDP by region and sector as well as EAC GDP by sector. Figure \ref{fig:wld_GDP_reg} shows Global GDP by region. The impact of the 2009 global financial crisis is clearly visible and GDP also has declined in 2015. % \todo{why?}. 
According to this data EAC GDP at basic prices has increased both in absolute value from 43.6 billion USD in 2005 to 101.6 billion USD in 2015, and as a share of global GDP from 0.096\% in 2005 to 0.137\% in 2015. \newline

Figure \ref{fig:wld_GDP_sec} shows global GDP by sector. In 2015, 25\% of global GDP was produced by financial and business services (FIB), followed by the education, health and other services category at 11.7\%. Agriculture and Fishing together only accounted for 4.2\% of global GDP in 2015.


%\todo[inline]{Add World Bank GDP line.}


\begin{figure}[h!]
\centering
\caption{\label{fig:wld_GDP_reg}\textsc{Global GDP by Region}}
\small{\textit{Millions of current USD at Basic Prices}}
\includegraphics[width=1\textwidth, trim= {0 0 0 0}, clip]{"../Figures/global_GDP_region".pdf} %trim={<left> <lower> <right> <upper>}
\end{figure}
\FloatBarrier

\begin{figure}[h!]
\centering
\caption{\label{fig:wld_GDP_sec}\textsc{Global GDP by Sector}}
\small{\textit{Millions of current USD at Basic Prices}}
\includegraphics[width=1\textwidth, trim= {0 0 0 0}, clip]{"../Figures/global_GDP_sector".pdf} %trim={<left> <lower> <right> <upper>}
\end{figure}
\FloatBarrier

Figure \ref{fig:EAC_GDP_sec} shows EAC GDP by sector. Here discrepancies between this harmonized data and the real world are very visible. By 2015, agricultural value added in Uganda was still around 30\% of GDP, whereas it is blow 20\% of GDP in the EORA data. The level of GDP at basic prices seems to be broadly in line, as GDP was around 27 billion USD at current prices in 2015, up from 9 Billion in 2005. The growth path however seems to be too strong in the years 2005-2008, and too flat from 2009-2015 compared to the real trajectory. Of the other countries, apart from moderate mismatches in sectoral value added shares for all EAC countries, there seems to be a major problem with the data for Tanzania. Tanzanian GDP was estimated at 18.4 Billion in 2005 and increased to 47.4 Billion in 2015. The EORA data show an initial GDP for Tanzania of 10.5 Billion at basic prices in 2005, which declines over the sample period to 8.5 Billion at basic prices. \newline



\begin{figure}[h!]
\centering
\caption{\label{fig:EAC_GDP_sec}\textsc{EAC GDP by Sector}}
\includegraphics[width=1\textwidth, trim= {0 0 0 0}, clip]{"../Figures/EAC_GDP_sector".pdf} %trim={<left> <lower> <right> <upper>}
\end{figure}
\FloatBarrier

Gross exports of EAC countries are shown in Figure \ref{fig:exp}. Here the level of Tanzanian exports is more in line with the level recorded by the World Bank. In terms of composition, it is evident that Uganda focuses on agricultural exports, comprising 38\% of exports over the analyzed period, while Rwanda has a disproportionate share in mining exports of about 24\%. The other EAC countries have a more balanced export mix, with Tanzania and Kenya also maintaining shares of 24\% and 29\%, respectively, in agriculture. 

% \todo[inline]{Add World Bank Exports line.}

\begin{figure}[h!]
\centering
\caption{\label{fig:exp}\textsc{EAC Gross Exports}}
\includegraphics[width=1\textwidth, trim= {0 0 0 0}, clip]{"../Figures/exports_stacked_ts".pdf} %trim={<left> <lower> <right> <upper>}
\end{figure}
\FloatBarrier

All of this of course strongly calls into question the reliability of this data to analyze developments in the EAC. The creators of this database write:

\begin{quote}
The current Eora tables that have been constructed with emphasis on a) representing large data items and b) fulfilling balancing conditions for large countries.

The goal of Eora is to to make a consistent global model. When smaller or developing economies have inconsistent or missing data the tables for these countries can become distorted during the process of building a consistent global model. %These problems can be identified using the Table Balancing Check reports. In some cases countries raw macroeonomic data is highly unreliable or conflicting. This can occur especially during periods of war, government transition, or hyperinflation. Eora is built by combining and reconciling various data sources, but these processes are purely algorithmic; there is no manual intervention in the raw data. Thus missing, incomplete, and conflicting raw data can mean that we are unable to realize a consistent, balanced, IO table for a country in a given year.
\end{quote}

\subsection*{EORA 2021 Revision}

The 2021 revision of EORA, incorporating updated administrative data from 2016 through 2018, and forecasts based on the World Economic Outlook through 2021, appears to better reflect macroeconomic totals for the EAC, but also introduces major changes and structural breaks in GVC indicators, some of which appear highly unrealistic. 

\begin{figure}[h!]
\centering
\caption{\label{fig:EAC_GDP_sec_21}\textsc{EAC GDP by Sector: EORA 2021}}
\includegraphics[width=1\textwidth, trim= {0 0 0 0}, clip]{"../Figures/EAC_GDP_sector_21".pdf} %trim={<left> <lower> <right> <upper>}
\end{figure}
\FloatBarrier


Figure \ref{fig:exp21} shows gross exports in the extended database. The total export volumes appear closer to the true values (in 2018 according to World Bank Data Uganda exported \$3B, Tanzania \$4B, and Kenya \$6B in current prices), but the agricultural content of exports is overstated. Uganda for example had more than 40\% of exports in mining in 2018. 

\begin{figure}[h!]
\centering
\caption{\label{fig:exp21}\textsc{EAC Gross Exports: EORA 2021}}
\includegraphics[width=1\textwidth, trim= {0 0 0 0}, clip]{"../Figures/exports_stacked_ts_21".pdf} %trim={<left> <lower> <right> <upper>}
\end{figure}
\FloatBarrier

Figure \ref{fig:outshares_ag_ts_21} shows gross-flows metrics. These also show some large structural breaks, especially for Tanzania and Burundi, bot no change to the overall story that Uganda and Kenya have significant export shares with the EAC, and Uganda and Rwanda have significant import shares with the EAC, while other countries EAC engagement remains low. 

\begin{figure}[h!]
\centering
\caption{\label{fig:outshares_ag_ts_21}\textsc{Decomposition of Output and Exports: EORA 2021}}
\includegraphics[width=1\textwidth, trim= {0 0 0 0}, clip]{"../Figures/output_shares_ag_ts_21".pdf} %trim={<left> <lower> <right> <upper>}
\end{figure}
\FloatBarrier

Figure \ref{fig:VSag_ts_21} shows measures of forward and backwards GVC integration with the updated EORA database. Compared to the pre-revision data, forward integration (E2R) is higher and backward integration (I2E) is lower. It is notable that the high I2E in Tanzania seems to be have been an artefact of the old data which vanishes in the opdated data. The updated data however are also a source of problems. It is highly inrealistic that Burundi has a high level of backwards integration at 50\% of exports being imported. 

\begin{figure}[h!]
\centering
\caption{\label{fig:VSag_ts_21}\textsc{GVC Integration of EAC Members: Aggregate: EORA 2021}}
\includegraphics[width=1\textwidth, trim= {0 0 0 0}, clip]{"../Figures/VS_ag_ts_21".pdf} %trim={<left> <lower> <right> <upper>}
\end{figure}
\FloatBarrier

Figure \ref{fig:KWW_fill_ts_21} also shows the more detailed KWW decomposition with the KWW data, with similar results to Figure \ref{fig:VSag_ts_21}. The upstreamness and downstreamness ratios are computed from this data following equations \ref{eq:US} and \ref{eq:DS}, and displayed in Figure \ref{fig:UP_DOWN_ag_ts_21}.


\begin{figure}[h!]
\centering
\caption{\label{fig:KWW_fill_ts_21}\textsc{KWW Decomposition of Gross Exports: EORA 2021}}
\includegraphics[width=1\textwidth, trim= {0 0 0 0}, clip]{"../Figures/KWW_fill_ts_21".pdf} %trim={<left> <lower> <right> <upper>}
\end{figure}
\FloatBarrier

Figure \ref{fig:UP_DOWN_ag_ts_21} shows a structural break, but not really a trend reversal, in the upstreamness and downstreamness ratios. 

\begin{figure}[h!] % \vspace{-0.4cm}
\centering
\caption{\label{fig:UP_DOWN_ag_ts_21}\textsc{Upstreamness and Downstreamness Ratios: EORA 2021}}
\includegraphics[width=1\textwidth, trim= {0 0 0 0}, clip]{"../Figures/UP_DOWN_ag_ts_21".pdf} %trim={<left> <lower> <right> <upper>}
% \vspace{-1cm}
\end{figure} 
\FloatBarrier 


\subsection*{Additional Tables and Figures}

\begin{figure}[!h]
\centering
\vspace{-2cm}
\caption{\label{fig:outshares}\textsc{Decomposition of Sectoral Output and Exports}}
\vspace*{\fill}
\begin{adjustbox}{center}
\includegraphics[width=1.75\textwidth, angle =270, trim= {1cm 0 0 0}, clip]{"../Figures/output_shares".pdf} %trim={<left> <lower> <right> <upper>}
\end{adjustbox}
\vspace*{\fill}
\end{figure}
\FloatBarrier

\begin{figure}[h!]
\centering
\caption{\label{fig:wldVB}\textsc{Aggregated Value Added Share Matrix (\textbf{VB}) 2015}}
\small{\textit{Shares in Percentage Terms, Columns Sum to 100 Percent}}
\includegraphics[width=1\textwidth, trim= {0 0 0 0}, clip]{"../Figures/heatmap_AG_VB".pdf} %trim={<left> <lower> <right> <upper>}
\end{figure}
\FloatBarrier

\begin{figure}[h!]
\centering
\caption{\label{fig:eacVB}\textsc{Disaggregated Value Added Share Tables: EAC in 2015}}
\small{\textit{Shares in Percentage Terms, Columns Sum to 100 Percent}}
\includegraphics[width=1\textwidth, trim= {0 0 0 0}, clip]{"../Figures/heatmap_VB_AG_EAC_tot".pdf} %trim={<left> <lower> <right> <upper>}
\end{figure}
\FloatBarrier

\begin{figure}[h!]
\centering
\caption{\label{fig:VAwld}\textsc{Aggregated MRIO Table in VA Terms: EAC and World Regions}}
\small{\textit{Millions of 2015 USD at Basic Prices on a Log10 Scale}}
\includegraphics[width=1\textwidth, trim= {0 0 0 0}, clip]{"../Figures/heatmap_VA_AG".pdf} %trim={<left> <lower> <right> <upper>}
\end{figure}
\FloatBarrier

\begin{figure}[h!]
\centering
\caption{\label{fig:VAexp}\textsc{EAC Domestic Value Added in Global Exports}}
\includegraphics[width=1\textwidth, trim= {0 0 0 0}, clip]{"../Figures/VA_exports_stacked_ts".pdf} %trim={<left> <lower> <right> <upper>}
\end{figure}
\FloatBarrier

\begin{figure}[h!]
\centering
\caption{\label{fig:VS}\textsc{GVC Integration of EAC Members: Sector Level: 2015}}
\includegraphics[width=1\textwidth, trim= {0 0 0 0}, clip]{"../Figures/VS".pdf} %trim={<left> <lower> <right> <upper>}
\end{figure}
\FloatBarrier

\begin{figure}[h!]
\centering
\caption{\label{fig:VSgr}\textsc{GVC Integration of EAC Members: Annual Growth 2005-2015}}
\includegraphics[width=1\textwidth, trim= {0 0 0 0}, clip]{"../Figures/VS_growth".pdf} %trim={<left> <lower> <right> <upper>}
\end{figure}
\FloatBarrier


\begin{figure}[h!] \vspace{-1cm}
\centering
\caption{\label{fig:KWW_fill_ts_EAC}\textsc{KWW Decomposition of Gross Exports to the EAC}}
\includegraphics[width=1\textwidth, trim= {0 0 0 0}, clip]{"../Figures/KWW_fill_ts_EAC".pdf} %trim={<left> <lower> <right> <upper>}
\vspace{-0.8cm}
\end{figure}
\FloatBarrier

\begin{figure}[h!] \vspace{-0.1cm}
\centering
\caption{\label{fig:KWW_fill_sec}\textsc{KWW Decomposition of Sector-Level Gross Exports in 2015}}
\includegraphics[width=1\textwidth, trim= {0 0 0 0}, clip]{"../Figures/KWW_fill_sec".pdf} %trim={<left> <lower> <right> <upper>}
\vspace{-1.5cm}
\end{figure}
\FloatBarrier


\begin{figure}[h!]
\centering
\caption{\label{fig:NRCA_IEAC}\textsc{NRCA for Inner-EAC Trade}}
\includegraphics[width=1\textwidth, trim= {0 0 0 0}, clip]{"../Figures/NRCA_IEAC".pdf} %trim={<left> <lower> <right> <upper>}
\end{figure}
\FloatBarrier


\end{document}
